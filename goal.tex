% 両面印刷する場合は `openany' を削除する
\documentclass[openany,11pt,papersize]{jsbook}

%パッケージの読み込みなど
% 報告書提出用スタイルファイル
\usepackage[final]{funpro}%最終報告書
%\usepackage[middle]{funpro}%中間報告書

% 画像ファイル (EPS, EPDF, PNG) を読み込むために
\usepackage[dvipdfmx]{graphicx,color}

%数式の表示に利用するため
\usepackage{amsmath,amssymb}

%アルゴリズムの表示に利用するパッケージ
\usepackage{algorithm}
\usepackage{algorithmic}

%枠をつけるためのパッケージ
\usepackage{ascmac}

%図の位置調整パッケージ
\usepackage{here}

%付録を作成するためのパッケージ
\usepackage{appendix}

%ドキュメント管理用パッケージ
\usepackage{docmute}


% ここから -->
\usepackage{calc,ifthen}
\newcounter{hoge}
\newcommand{\fake}[1]{\whiledo{\thehoge<70}{#1\stepcounter{hoge}}%
  \setcounter{hoge}{0}}
% <-- ここまで 削除してもよい


% 年度の指定
\thisYear{2016}

% プロジェクト名
\jProjectName{FUN-ECM プロジェクト}

% [簡易版のプロジェクト名]{正式なプロジェクト名}
% 欧文のプロジェクト名が極端に長い(2行を超える)場合は,短い記述を
% 任意引数として渡す.
%\eProjectName[Making Delicious curry]{How to make delicious curry of Hakodate}
\eProjectName{FUN-ECM Project}


% <プロジェクト番号>-<グループ名>
\ProjectNumber{15-A}

% グループ名
\jGroupName{Aグループ}
\eGroupName{A Group}

% プロジェクトリーダ
\ProjectLeader{1014129}{池野竜將}{Ryusuke Ikeno}

% グループリーダ
\GroupLeader  {1014129}{池野竜將}{Ryusuke Ikeno}

% メンバー数
\SumOfMembers{8}
% グループメンバ
\GroupMember  {1}{1014068}{駒ヶ嶺壮}{Sou Komagamine}
\GroupMember  {2}{1014109}{伊藤有輝}{Yuki Ito}
\GroupMember  {3}{1014129}{池野竜將}{Ryusuke Ikeno}
\GroupMember  {4}{1014137}{千葉大樹}{Daiju Chiba}
\GroupMember  {5}{1014164}{橋本和典}{Kazunori Hashimoto}
\GroupMember  {6}{1014168}{山下哲平}{Teppei Yamashita}
\GroupMember  {7}{1014209}{源啓多}{Keita Minamoto}
\GroupMember  {8}{1013150}{亀谷浩也}{Hiroya Kametani}

% 指導教員
\jadvisor{白勢政明,由良文孝}
% 複数人数いる場合はカンマ(,)で区切る.カンマの前後に空白は入れない.
\eadvisor{Masaaki Shirase, Fumitaka Yura}

% 論文提出日
\jdate{2016年7月27日}
\edate{July~27, 2016}


\begin{document}

\chapter{到達目標}

\section{本プロジェクトにおける目的}\label{sec:mokuteki}

FUN-ECMがECM-NETにランクインするためには去年のプログラムをより改善する必要がある.この目標を達成するにあたって,2つの目標を立てることにした.

\bunseki{伊藤有輝}

\subsection{プログラムの高速化}\label{sec:goal1}

ECM-NETにランクインするためには,巨大な素因数を発見しなければならない.巨大な素因数を発見するためには桁数の大きい合成数を素因数分解する必要があるが,それには多大な時間がかかってしまう.また,ECMは一度の試行で素因数を必ず発見できるとは限らず,数千回程度の試行が必要となる.そのためプログラムの処理を効率の良いアルゴリズムに変更し,処理を高速化させる必要がある.この目標を達成するにあたって,2つの目標を立てることにした.

\begin{itemize}
\item 昨年度のプログラムのアルゴリズムの理解
\item 昨年度のプログラムの書き換えたものの実装
\end{itemize}

まず,昨年度のプログラムを高速化するにはアルゴリズムの理解が必要である.また,楕円曲線法では大学までの学習で使用していない数学の概念を使用するため,基礎学習を行う.

\bunseki{伊藤有輝}

\subsection{FUN-ECMの活動発信}\label{sec:goal2}

今年度では,ただランクインを目指すだけでなく,函館から楕円曲線,素因数分解の重要性について発信することに決め,ホームページを設立することとした.

\bunseki{伊藤有輝}

\section{課題達成の為の班分け}
前年度のプロジェクトでは前期で楕円曲線法についての学習を行い,後期でアルゴリズムの提案・実装を行っていた.しかし,このような日程でプロジェクトを進行していくと以下のような問題が発生した.

\begin{itemize}
\item 実際にプログラムを実装する期間が少ない
\item 完成したプログラムを試行する期間が少ない
\item 巨大な合成数の分解を行いにくい
\end{itemize}

そのため,本プロジェクトでは5月中旬まで全員で最低限の基礎学習を行い,そこから理論班とプログラミング班の2つに分けて作業を行うこととした.また,後期には広報班を作成し,3つの作業を並行で行うこととした.以下にそれぞれの班の課題について述べる.

\begin{description}
 \item[理論班]\mbox{}\\ 
	    ECMについて理解を深め,高速化の新たなアルゴリズムを提案する.
 \item[プログラミング班]\mbox{}\\
	    基礎学習や理論班がまとめたアルゴリズムを実装し高速化を行う.
 \item[広報班]\mbox{}\\
	    ECMについて理解してもらえるようなWebページの作成をする.
\end{description}
\bunseki{伊藤有輝}

\end{document}