% 両面印刷する場合は `openany' を削除する
\documentclass[openany,11pt,papersize]{jsbook}

%パッケージの読み込みなど
% 報告書提出用スタイルファイル
\usepackage[final]{funpro}%最終報告書
%\usepackage[middle]{funpro}%中間報告書

% 画像ファイル (EPS, EPDF, PNG) を読み込むために
\usepackage[dvipdfmx]{graphicx,color}

%数式の表示に利用するため
\usepackage{amsmath,amssymb}

%アルゴリズムの表示に利用するパッケージ
\usepackage{algorithm}
\usepackage{algorithmic}

%枠をつけるためのパッケージ
\usepackage{ascmac}

%図の位置調整パッケージ
\usepackage{here}

%付録を作成するためのパッケージ
\usepackage{appendix}

%ドキュメント管理用パッケージ
\usepackage{docmute}

% ここから -->
\usepackage{calc,ifthen}
\newcounter{hoge}
\newcommand{\fake}[1]{\whiledo{\thehoge<70}{#1\stepcounter{hoge}}%
  \setcounter{hoge}{0}}
% <-- ここまで 削除してもよい


% 年度の指定
\thisYear{2016}

% プロジェクト名
\jProjectName{FUN-ECM プロジェクト}

% [簡易版のプロジェクト名]{正式なプロジェクト名}
% 欧文のプロジェクト名が極端に長い(2行を超える)場合は,短い記述を
% 任意引数として渡す.
%\eProjectName[Making Delicious curry]{How to make delicious curry of Hakodate}
\eProjectName{FUN-ECM Project}


% <プロジェクト番号>-<グループ名>
\ProjectNumber{15-A}

% グループ名
\jGroupName{Aグループ}
\eGroupName{A Group}

% プロジェクトリーダ
\ProjectLeader{1014129}{池野竜將}{Ryusuke Ikeno}

% グループリーダ
\GroupLeader  {1014129}{池野竜將}{Ryusuke Ikeno}

% メンバー数
\SumOfMembers{8}
% グループメンバ
\GroupMember  {1}{1014068}{駒ヶ嶺壮}{Sou Komagamine}
\GroupMember  {2}{1014109}{伊藤有輝}{Yuki Ito}
\GroupMember  {3}{1014129}{池野竜將}{Ryusuke Ikeno}
\GroupMember  {4}{1014137}{千葉大樹}{Daiju Chiba}
\GroupMember  {5}{1014164}{橋本和典}{Kazunori Hashimoto}
\GroupMember  {6}{1014168}{山下哲平}{Teppei Yamashita}
\GroupMember  {7}{1014209}{源啓多}{Keita Minamoto}
\GroupMember  {8}{1013150}{亀谷浩也}{Hiroya Kametani}

% 指導教員
\jadvisor{白勢政明,由良文孝}
% 複数人数いる場合はカンマ(,)で区切る.カンマの前後に空白は入れない.
\eadvisor{Masaaki Shirase, Fumitaka Yura}

% 論文提出日
\jdate{2016年7月27日}
\edate{July~27, 2016}


\begin{document}

\chapter{後期活動成果}

\section{理論班}
検証するにあたり,まず今年度はどのような方法で昨年度のプログラムと比較するか考えた.昨年度の報告書を参考にし,昨年度では素数を入力しアルゴリズムが終了するまで時間を計測し比較していたが,今年度はプログラムの処理速度の改善ではなく素因数を発見する確率を上げたため今回はこの方法では検証しなかった.おそして,白勢先生にアドバイスをいただき,そのアドバイスに基づいて,検証を行った.統計に関しては検証に時間がかかり検証回数が少なかったので統計としてはデータが少なく信頼性が低いが,数値として結果を表すことができた.
\bunseki{橋本和典}

\section{プログラム班}
後期の活動では,新たに3つのアルゴリズムを導入した.改善率などの具体的な数値に関しては\ref{sec:theoryresult}に示しているので省略する.また昨年度までは実装されていなかったStage2を新たに適用した.理論班の検証結果によるとECMプログラムを最大で15\% ほど高速化することに成功した.また,広報班と協力しECMプログラムの使用法やアルゴリズムについて解説した.また,前期に引き続き,巨大な合成数の分解を行った.後期の間に発見されたもっとも大きな素因数は95371895138956317843189468149739281(35桁)であり,前期に比べてより大きな素因数を見つけることができるように改善された.素因数が見つかった際のログを以下に示す.
\begin{itembox}[H]{30桁発見の際のログ}
Stage1: d = 575307328912030177342905303494973165817419571029265956994954365680489

Stage1 time: 3.971019 seconds

Stage2 time: ---

total time: 3.971 seconds

Y=2085933894638188510940267585725531773871162772209879420070663851503610

--------------------------------------------------

@ probable prime factor found: 95371895138956317843189468149739281  digits: 35 cofactor: 21974831956736523892809147362583287
\end{itembox}

\section{広報班}
後期の半期間で我々の活動内容やECMの基礎理論について紹介するwebページを作成した.ページ自体は期間内に完成させることが出来たが,2か月半という短い時間での制作だったため最終発表までに,作成したページに関するアンケートをとることが出来なかった.そのため,ページのコンテンツ力が高いか否かについての統計をとることが出来ず残念ながら効果の測定を十分に行うことができなかった.十分な測定とは言えないが,最終発表内の時間を用いてページを紹介し,有志でアンケートを依頼した結果若干名ではあるが回答してくれた.回答してくれた方の約半数がECMの基礎理論について理解できたと回答してくれた.

\bunseki{橋本和典}

\end{document}