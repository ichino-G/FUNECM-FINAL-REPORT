% 両面印刷する場合は `openany' を削除する
\documentclass[openany,11pt,papersize]{jsbook}

%パッケージの読み込みなど
% 報告書提出用スタイルファイル
\usepackage[final]{funpro}%最終報告書
%\usepackage[middle]{funpro}%中間報告書

% 画像ファイル (EPS, EPDF, PNG) を読み込むために
\usepackage[dvipdfmx]{graphicx,color}

%数式の表示に利用するため
\usepackage{amsmath,amssymb}

%アルゴリズムの表示に利用するパッケージ
\usepackage{algorithm}
\usepackage{algorithmic}

%枠をつけるためのパッケージ
\usepackage{ascmac}

%図の位置調整パッケージ
\usepackage{here}

%付録を作成するためのパッケージ
\usepackage{appendix}

%ドキュメント管理用パッケージ
\usepackage{docmute}


% ここから -->
\usepackage{calc,ifthen}
\newcounter{hoge}
\newcommand{\fake}[1]{\whiledo{\thehoge<70}{#1\stepcounter{hoge}}%
  \setcounter{hoge}{0}}
% <-- ここまで 削除してもよい


% 年度の指定
\thisYear{2016}

% プロジェクト名
\jProjectName{FUN-ECM プロジェクト}

% [簡易版のプロジェクト名]{正式なプロジェクト名}
% 欧文のプロジェクト名が極端に長い(2行を超える)場合は,短い記述を
% 任意引数として渡す.
%\eProjectName[Making Delicious curry]{How to make delicious curry of Hakodate}
\eProjectName{FUN-ECM Project}


% <プロジェクト番号>-<グループ名>
\ProjectNumber{15-A}

% グループ名
\jGroupName{Aグループ}
\eGroupName{A Group}

% プロジェクトリーダ
\ProjectLeader{1014129}{池野竜將}{Ryusuke Ikeno}

% グループリーダ
\GroupLeader  {1014129}{池野竜將}{Ryusuke Ikeno}

% メンバー数
\SumOfMembers{8}
% グループメンバ
\GroupMember  {1}{1014068}{駒ヶ嶺壮}{Sou Komagamine}
\GroupMember  {2}{1014109}{伊藤有輝}{Yuki Ito}
\GroupMember  {3}{1014129}{池野竜將}{Ryusuke Ikeno}
\GroupMember  {4}{1014137}{千葉大樹}{Daiju Chiba}
\GroupMember  {5}{1014164}{橋本和典}{Kazunori Hashimoto}
\GroupMember  {6}{1014168}{山下哲平}{Teppei Yamashita}
\GroupMember  {7}{1014209}{源啓多}{Keita Minamoto}
\GroupMember  {8}{1013150}{亀谷浩也}{Hiroya Kametani}

% 指導教員
\jadvisor{白勢政明,由良文孝}
% 複数人数いる場合はカンマ(,)で区切る.カンマの前後に空白は入れない.
\eadvisor{Masaaki Shirase, Fumitaka Yura}

% 論文提出日
\jdate{2016年7月27日}
\edate{July~27, 2016}


\begin{document}

\chapter{プロジェクト内のインターワーキング}
\begin{itemize}
\item 池野竜將(プロジェクトリーダー・プログラミング班)
 \begin{enumerate}
 \renewcommand{\labelenumi}{(\arabic{enumi})}
 \item 楕円曲線法の基礎を学んだ.
 \item 大まかな作業スケジュールを作成し,進捗管理を行った.
 \item 源と協力して前年度のECMプログラムを理解した.
 \item 源のコーディング作業にアドバイスをした.
 \item 理論班からのプログラミング班に関しての質問に回答し,必要があれば聞かれた内容を源に伝えた.
 \item 中間発表会に向けて,プレゼンテーション資料・原稿の原案を作成した.
 \item 中間発表会に向けて,プログラミング班の部分のプレゼンテーション資料を作成した.
 \item 成果発表会に向けて,プログラミング班の部分のプレゼンテーション資料を作成した.
 \end{enumerate}
 
\item 源啓多(プログラミング班)
 \begin{enumerate}
 \renewcommand{\labelenumi}{(\arabic{enumi})}
 \item 楕円曲線法の基礎を学んだ.
 \item 池野と協力して前年度のECMプログラムを理解した.
 \item ECMプログラムのバージョン管理の為,Gitを学んだ.
 \item 前年度のECMプログラムの実装上のミス(\ref{sec:alg1})を改善した.
 \item ECMプログラム改善のために,新たなアルゴリズム(\ref{sec:alg2}, \ref{sec:alg3})の実装を行った.
 \item 中間発表会に向けて,プログラミング班のプレゼンテーション資料・原稿を作成した.
 \item Stage2の解読・実装をいち早く進めた.
 \item 解析班の作業を助けるためのマクロを作成した.
 \item 広報班に協力し,ウェブページの作成の手助けをした.
 \item 成果発表会に向けて,プログラミング班の部分のプレゼンテーション資料の修正を行った.
 \item 最終報告書作成の際,各グループの作成物のレビューを行った.
 \end{enumerate}
 
\item 山下哲平(理論班)
 \begin{enumerate}
 \renewcommand{\labelenumi}{(\arabic{enumi})}
 \item 楕円曲線法の基礎を学んだ.
 \item 伊藤・駒ヶ嶺と協力してEdwards Curveを利用したECMアルゴリズムの読解を行い,プログラミング班に提案を行った.
 \item 伊藤・駒ヶ嶺と協力してAtkin-Morain ECPPアルゴリズムの理解に取り組んだ.
 \item 中間発表会に向けて,「背景」の部分についてポスターをを作成した.
 \item プログラミング班の要請でプログラムの速度について簡易的な検証を行った.
 \item 伊藤と協力してウェブページの基本的な要素を作成した.
 \item 最終報告書の広報班ページを作成する際に,中心なって活動し進捗を管理した.
 \end{enumerate}
 
\item 伊藤有輝(理論班)
 \begin{enumerate}
 \renewcommand{\labelenumi}{(\arabic{enumi})}
 \item 楕円曲線法の基礎を学んだ.
 \item 駒ヶ嶺と協力して,エドワーズ曲線の式が導き出される過程を学んだ.
 \item 駒ヶ嶺・山下と協力し,Edwards Curveを利用したECMアルゴリズムの読解を行った.
 \item 駒ヶ嶺・山下と協力し,Atkin-Morain ECPPアルゴリズムの理解に取り組み,プログラミング班に提案を行った.
 \item 中間発表会に向けて,「理論班」の部分のプレゼンテーション資料を作成した.
 \item 源・池野と協力し,Githubの使い方を理解して広報班に伝えた.
 \item 成果発表会に向けて,理論班の部分のプレゼンテーション資料の修正を行った.
 \item 山下と協力してウェブページの基本的な要素を作成した.

 \end{enumerate}
 
\item 駒ヶ嶺壮(理論班)
 \begin{enumerate}
 \renewcommand{\labelenumi}{(\arabic{enumi})}
 \item 楕円曲線法の基礎を学んだ.
 \item 伊藤と協力して,エドワーズ曲線の式が導き出される過程を学んだ.
 \item 山下・伊藤と協力してEdwards Curveを利用したECMアルゴリズムの読解を行った.
 \item 山下・伊藤と協力してAtkin-Morain ECPPアルゴリズムの理解に取り組んだ.
 \item 中間発表会に向けて,「理論班」の部分のポスターを作成した.
 \item 広報班のウェブページ作成のため,過去の作業ログを見直し,まとめた.
 \item 中間報告書の間違いを修正し,新たにセクションを追加した.
 \end{enumerate}
 
\item 橋本和典(理論班)
 \begin{enumerate}
 \renewcommand{\labelenumi}{(\arabic{enumi})}
 \item 楕円曲線法の基礎を学んだ.
 \item 千葉・亀谷と協力して入門書を読み,基礎学習を行った.
 \item 亀谷と協力して基礎学習を簡潔にまとめた解説ノートを作成した.
 \item 中間発表会に向けて,千葉・亀谷と協力して来るであろう質問を予測して対策を行った.
 \item ECMの改善に直結するような文献を探した。
 \item 中間発表会に向けて、ポスターの「理論班」の章を英訳した。
 \item 理論班で検証を行う際に,管理者として中心となって作業した.
 \item 行った検証の結果をまとめ,グラフ化して見やすくした.
 \end{enumerate}
 
\item 千葉大樹(理論班)
 \begin{enumerate}
 \renewcommand{\labelenumi}{(\arabic{enumi})}
 \item 楕円曲線法の基礎を学んだ.
 \item 亀谷・橋本と協力して入門書を読み,基礎学習を行った.
 \item 中間発表会に向けて,ECMについての英論文から重要な単語を抜粋し解説した.
 \item 中間発表会に向けて,亀谷・橋本と協力して来るであろう質問を予測して対策を行った.
 \item 中間発表会に向けて、ポスターの「プログラミング班」の章を英訳した。
 \item 理論班で検証を行う際に,実際にプログラムを動かし,データを全体に共有した.
 \end{enumerate}
 
\item 亀谷浩也(理論班)
 \begin{enumerate}
 \renewcommand{\labelenumi}{(\arabic{enumi})}
 \item 楕円曲線法の基礎を学んだ.
 \item 橋本・千葉と協力して入門書を読み,基礎学習を行った.
 \item 橋本と協力して,基礎学習を簡潔にまとめた解説ノートを作成した.
 \item 中間発表会に向けて,橋本・千葉と協力して来るであろう質問を予測して対策を行った.
 \item 中間発表会に向けて、ポスターの「概要・基礎学習」の章を英訳した。
 \item 理論班で検証を行う際に,データの管理やまとめを手伝い,橋本の補佐として活動した.
 \item 最終報告書の広報班ページを作成する際に,中心なって活動し進捗を管理した.
 \end{enumerate}
 
\end{itemize}

\bunseki{池野竜將}

\end{document}
