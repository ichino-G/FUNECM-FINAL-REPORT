% 両面印刷する場合は `openany' を削除する
\documentclass[openany,11pt,papersize]{jsbook}

%パッケージの読み込みなど
% 報告書提出用スタイルファイル
\usepackage[final]{funpro}%最終報告書
%\usepackage[middle]{funpro}%中間報告書

% 画像ファイル (EPS, EPDF, PNG) を読み込むために
\usepackage[dvipdfmx]{graphicx,color}

%数式の表示に利用するため
\usepackage{amsmath,amssymb}

%アルゴリズムの表示に利用するパッケージ
\usepackage{algorithm}
\usepackage{algorithmic}

%枠をつけるためのパッケージ
\usepackage{ascmac}

%図の位置調整パッケージ
\usepackage{here}

%付録を作成するためのパッケージ
\usepackage{appendix}

%ドキュメント管理用パッケージ
\usepackage{docmute}

% ここから -->
\usepackage{calc,ifthen}
\newcounter{hoge}
\newcommand{\fake}[1]{\whiledo{\thehoge<70}{#1\stepcounter{hoge}}%
  \setcounter{hoge}{0}}
% <-- ここまで 削除してもよい


% 年度の指定
\thisYear{2016}

% プロジェクト名
\jProjectName{FUN-ECM プロジェクト}

% [簡易版のプロジェクト名]{正式なプロジェクト名}
% 欧文のプロジェクト名が極端に長い(2行を超える)場合は,短い記述を
% 任意引数として渡す.
%\eProjectName[Making Delicious curry]{How to make delicious curry of Hakodate}
\eProjectName{FUN-ECM Project}


% <プロジェクト番号>-<グループ名>
\ProjectNumber{15-A}

% グループ名
\jGroupName{Aグループ}
\eGroupName{A Group}

% プロジェクトリーダ
\ProjectLeader{1014129}{池野竜將}{Ryusuke Ikeno}

% グループリーダ
\GroupLeader  {1014129}{池野竜將}{Ryusuke Ikeno}

% メンバー数
\SumOfMembers{8}
% グループメンバ
\GroupMember  {1}{1014068}{駒ヶ嶺壮}{Sou Komagamine}
\GroupMember  {2}{1014109}{伊藤有輝}{Yuki Ito}
\GroupMember  {3}{1014129}{池野竜將}{Ryusuke Ikeno}
\GroupMember  {4}{1014137}{千葉大樹}{Daiju Chiba}
\GroupMember  {5}{1014164}{橋本和典}{Kazunori Hashimoto}
\GroupMember  {6}{1014168}{山下哲平}{Teppei Yamashita}
\GroupMember  {7}{1014209}{源啓多}{Keita Minamoto}
\GroupMember  {8}{1013150}{亀谷浩也}{Hiroya Kametani}

% 指導教員
\jadvisor{白勢政明,由良文孝}
% 複数人数いる場合はカンマ(,)で区切る.カンマの前後に空白は入れない.
\eadvisor{Masaaki Shirase, Fumitaka Yura}

% 論文提出日
\jdate{2016年7月27日}
\edate{July~27, 2016}


\begin{document}

\chapter{まとめ}

\section{前期活動結果}
前期は参考資料,論文,担当教員の白勢先生の講義による楕円曲線法の理解から始め,楕円曲線が楕円曲線法においていつどのように使われるかを理解した.その後,理論班,プロジェクト班の2班に分かれ作業を行った.理論班は,論文,入門書の読解をし,プログラム高速化のための改善案を出すことに成功した.しかし,前期中にプログラミング班が実装することはできなかった.プログラミング班は前年度のプロジェクトで作成されたECMプログラムを理解した.その後,実装ミスの改善や,新たなアルゴリズムの実装を行い,計算コストの減少に成功した.

\bunseki{千葉大樹}

\section{後期の展望}

後期は,理論班が作成したAtkin-Morain ECPPアルゴリズムを実装し,さらにECMプログラムの改善を図る.また,大きな合成数の分解を続けECMNETへのランクインを目指す.加えて,前期中に活動できなかった広報について新たに班を設置し活動していく.

\bunseki{橋本和典}

\section{後期活動結果}
後期はプログラミング,理論,広報にわかれ作業を始めた.プログラム班は初めに,前期中に理論班によって提案されたAtkin-Morain ECPPの実装を行った.その後は,スカラー倍算の高速化アルゴリズムを実装し,プログラムの高速化に成功した.また新たな試みとして今まで実装されていなかったStage2の実装を行ったが,効率はほとんど変わらなかった.理論班は完成したプログラムを検証するために,検証方法の調査・提案を行った.その後,自分たちで提案した検証方法を元に検証を行い,結果をまとめた.広報班では,最初にウェブページのコンテンツやターゲットについての提案を行った.その結果,メインターゲットを未来大を中心とする情報系の大学生とし,ECMについての基礎理論についてのページを作成することにした.またサブターゲットとして,来年のプロジェクトメンバー向けに今年度作成したプログラムについてのページを作成することにした.完成したウェブページはgh-pagesというサービスを利用して公開した.
\bunseki{池野竜將}

\section{全体を通して}
ECMを利用した素因数分解プログラムは,検証の結果分解する合成数の桁数が大きければ大きいほど改善しており,最大で15\%ほど高速化されている.しかし,発見できた合成数は112544281755782732673671367061(30)が最大であり,ECMNETへのランクインには最低でも64桁以上の素因数を発見する必要があるため,現状ではECMNETへのランクインは難しい.また,今年度の新しい活動として行ったFUN-ECMの広報活動は,アンケートによると理解出来た人とできない人に分かれたが,アンケートの解答数が少なく,評価はできなかった.
\bunseki{橋本和典}

\section{今後の課題と展望}
\begin{itemize}
\item 今までの試行で発見できた合成数は30桁が最大であり,ECMNETへのランクインは難しいと予想される.しかし,ECMは運要素の強いアルゴリズムの為,どのような原因で素因数が発見できていないかを理解していない.そのため,来年度は既に分解されている合成数の分解を並行して行い,その時点のプログラムでどれくらいの桁数の素因数を発見ができるかについても確認・検証が必要だと考えられる.
\item 広報作業開始が後期であったため,作成後に評価をするための時間が十分に取れなかった.そのため,来年度も広報活動を行うのであれば,前期から活動を開始し,学生からのフィードバックでを受ける時間を確保することが必要だと考えられる.
\end{itemize}

\bunseki{池野竜將}
\end{document}