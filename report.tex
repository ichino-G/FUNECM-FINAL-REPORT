% プロジェクト学習中間報告書書式テンプレート ver.1.0 (iso-2022-jp)
% 両面印刷する場合は `openany' を削除する
\documentclass[openany,11pt,papersize]{jsbook}

%パッケージの読み込みなど
% 報告書提出用スタイルファイル
\usepackage[final]{funpro}%最終報告書
%\usepackage[middle]{funpro}%中間報告書

% 画像ファイル (EPS, EPDF, PNG) を読み込むために
\usepackage[dvipdfmx]{graphicx,color}

%数式の表示に利用するため
\usepackage{amsmath,amssymb}

%アルゴリズムの表示に利用するパッケージ
\usepackage{algorithm}
\usepackage{algorithmic}

%枠をつけるためのパッケージ
\usepackage{ascmac}

%図の位置調整パッケージ
\usepackage{here}

%付録を作成するためのパッケージ
\usepackage{appendix}

%ドキュメント管理用パッケージ
\usepackage{docmute}


% ここから -->
\usepackage{calc,ifthen}
\newcounter{hoge}
\newcommand{\fake}[1]{\whiledo{\thehoge<70}{#1\stepcounter{hoge}}%
  \setcounter{hoge}{0}}
% <-- ここまで 削除してもよい


% 年度の指定
\thisYear{2016}

% プロジェクト名
\jProjectName{FUN-ECM プロジェクト}

% [簡易版のプロジェクト名]{正式なプロジェクト名}
% 欧文のプロジェクト名が極端に長い(2行を超える)場合は,短い記述を
% 任意引数として渡す.
%\eProjectName[Making Delicious curry]{How to make delicious curry of Hakodate}
\eProjectName{FUN-ECM Project}


% <プロジェクト番号>-<グループ名>
\ProjectNumber{15-A}

% グループ名
\jGroupName{Aグループ}
\eGroupName{A Group}

% プロジェクトリーダ
\ProjectLeader{1014129}{池野竜將}{Ryusuke Ikeno}

% グループリーダ
\GroupLeader  {1014129}{池野竜將}{Ryusuke Ikeno}

% メンバー数
\SumOfMembers{8}
% グループメンバ
\GroupMember  {1}{1014068}{駒ヶ嶺壮}{Sou Komagamine}
\GroupMember  {2}{1014109}{伊藤有輝}{Yuki Ito}
\GroupMember  {3}{1014129}{池野竜將}{Ryusuke Ikeno}
\GroupMember  {4}{1014137}{千葉大樹}{Daiju Chiba}
\GroupMember  {5}{1014164}{橋本和典}{Kazunori Hashimoto}
\GroupMember  {6}{1014168}{山下哲平}{Teppei Yamashita}
\GroupMember  {7}{1014209}{源啓多}{Keita Minamoto}
\GroupMember  {8}{1013150}{亀谷浩也}{Hiroya Kametani}

% 指導教員
\jadvisor{白勢政明,由良文孝}
% 複数人数いる場合はカンマ(,)で区切る.カンマの前後に空白は入れない.
\eadvisor{Masaaki Shirase, Fumitaka Yura}

% 論文提出日
\jdate{2016年7月27日}
\edate{July~27, 2016}


\begin{document}

% 表紙
\maketitle

%前付け
\frontmatter

<<<<<<< HEAD
%概要
% 両面印刷する場合は `openany' を削除する
\documentclass[openany,11pt,papersize]{jsbook}

%パッケージの読み込みなど
% 報告書提出用スタイルファイル
\usepackage[final]{funpro}%最終報告書
%\usepackage[middle]{funpro}%中間報告書

% 画像ファイル (EPS, EPDF, PNG) を読み込むために
\usepackage[dvipdfmx]{graphicx,color}

%数式の表示に利用するため
\usepackage{amsmath,amssymb}

%アルゴリズムの表示に利用するパッケージ
\usepackage{algorithm}
\usepackage{algorithmic}

%枠をつけるためのパッケージ
\usepackage{ascmac}

%図の位置調整パッケージ
\usepackage{here}

%付録を作成するためのパッケージ
\usepackage{appendix}

%ドキュメント管理用パッケージ
\usepackage{docmute}


% ここから -->
\usepackage{calc,ifthen}
\newcounter{hoge}
\newcommand{\fake}[1]{\whiledo{\thehoge<70}{#1\stepcounter{hoge}}%
  \setcounter{hoge}{0}}
% <-- ここまで 削除してもよい


% 年度の指定
\thisYear{2016}

% プロジェクト名
\jProjectName{FUN-ECM プロジェクト}

% [簡易版のプロジェクト名]{正式なプロジェクト名}
% 欧文のプロジェクト名が極端に長い(2行を超える)場合は,短い記述を
% 任意引数として渡す.
%\eProjectName[Making Delicious curry]{How to make delicious curry of Hakodate}
\eProjectName{FUN-ECM Project}


% <プロジェクト番号>-<グループ名>
\ProjectNumber{15-A}

% グループ名
\jGroupName{Aグループ}
\eGroupName{A Group}

% プロジェクトリーダ
\ProjectLeader{1014129}{池野竜將}{Ryusuke Ikeno}

% グループリーダ
\GroupLeader  {1014129}{池野竜將}{Ryusuke Ikeno}

% メンバー数
\SumOfMembers{8}
% グループメンバ
\GroupMember  {1}{1014068}{駒ヶ嶺壮}{Sou Komagamine}
\GroupMember  {2}{1014109}{伊藤有輝}{Yuki Ito}
\GroupMember  {3}{1014129}{池野竜將}{Ryusuke Ikeno}
\GroupMember  {4}{1014137}{千葉大樹}{Daiju Chiba}
\GroupMember  {5}{1014164}{橋本和典}{Kazunori Hashimoto}
\GroupMember  {6}{1014168}{山下哲平}{Teppei Yamashita}
\GroupMember  {7}{1014209}{源啓多}{Keita Minamoto}
\GroupMember  {8}{1013150}{亀谷浩也}{Hiroya Kametani}

% 指導教員
\jadvisor{白勢政明,由良文孝}
% 複数人数いる場合はカンマ(,)で区切る.カンマの前後に空白は入れない.
\eadvisor{Masaaki Shirase, Fumitaka Yura}

% 論文提出日
\jdate{2016年7月27日}
\edate{July~27, 2016}


\begin{document}

% 和文概要
\begin{jabstract}
 私たちのプロジェクトの目的は,より大きな桁数の素因数を見つけることである.素因数分解は,約40年前から重要になってきている.その理由は,RSA暗号にある.RSA暗号は,安全性を2つの大きな素因数からなる合成数の素因数分解が難しいことに依存している.しかし,技術の発展とともに素因数分解が従来よりも容易になってきてしまっているため,RSA暗号が破られる可能性が高くなっている.そこで今注目されているのが楕円曲線暗号である.楕円曲線暗号は,RSA暗号と同じ鍵長で高い安全性を保障することができる.そこで私たちは素因数分解をより簡単なものとすることで,RSA暗号から楕円曲線暗号を主流とさせたい.
 私たちは,大きな素因数の発見のために,色々な文献を読んでその中から素因数分解を行うプログラムの改良法を発見する理論班と,それらの理論を利用して実際にプログラムの実装・改良を行い,プログラムを高速化させるプログラム班に分かれて活動を行った.
 理論班は,素因数分解がより高速に行われるようなアルゴリズムの発見を目標とした.楕円曲線法(ECM)のプログラムは点の加算の繰り返しで行われるため,加算の計算コストを減らすことで高速な計算を可能とするための活動を行った.Atkin Morain ECPPを利用することで,従来のECMよりも計算コストを削減できることを発見した.
 プログラム班は,前年度に作成された素因数分解プログラムをさらに高速化することを目標とした.前年度と同様に大きな数を扱うために,任意精度演算ライブラリのGMPを使用した.また,プログラムの並列実行を行うために,並列プログラムの為のAPIであるOpenMPを導入した.前年度に実装されたエドワーズ曲線よりも効率よく計算を行うため,extended twisted Edwards coordinatesを採用した.同じ合成数に対してプログラムを実行する際の因数の発見確率をあげるために,パラメータYの値をランダムに設定した.
 また,理論班とプログラム班で情報の交換を行ったり,協力を行ったりなど,2つの班の活動により,素因数分解を高速に行うことができるプログラムが完成した.
 更に,今年度からの活動としてより多くの人にECMについて知ってもらうため,私たちFUN-ECMの活動内容を広報することにした.広報の方法として新たに広報班を結成し,簡単に閲覧できるようにWebページを作成することにした.閲覧するターゲットは主に情報系の大学生とした.
 広報班の活動により,半期で設定したターゲットに向けたWebページを作成することができた.

% 和文キーワード
\begin{jkeyword}
素因数分解, 楕円曲線法, ECMNET, エドワーズ曲線, 射影座標, RSA暗号
\end{jkeyword}
\bunseki{山下哲平}
\end{jabstract}

%英語の概要

\begin{eabstract}

~The goal of our project team is to find prime factor as large as possible. 
Prime factorizations have become more important since about forty years ago 
because the difficulty of prime factorization is related to security of RSA cryptosystems 
which used for the Internet. 
However, prime factorization is getting to easier by development in technology. 
Therefore, security of RSA is less compared to previously. 
That's why Elliptic Curve Cryptography (ECC) is paid more attention than RSA now. 
ECC ensure security better than RSA cryptosystem with same key length. 
Accordingly, we make prime factorization simplify, 
we would like to change main cryptosystem from RSA cryptosystem to ECC. 

~We divided into two groups, one is \lq\lq theory group'' that reads various literature 
and find algorithms for prime factorizations to calculate faster, 
the other is \lq\lq programming group'' that write a program based on the algorithms.

 ~\lq\lq Theory group'' aims to find algorithm of prime factorizations to calculate faster. 
The ECM program repeats process of addition law many times over, therefore we reduced calculation. To access Atkin Morain construction, we were successful in calculation faster compared to previously.

~“Programming group” aims to improve program of last year project team faster than before. To treat large number likewise last year, we used arbitrary-precision arithmetic library called GMP. Also, we parallelize the program, we introduce Open MP which is API for parallel program. We implement extended twisted Edwards coordinates efficient than Edwards curve implemented last year. Also, we set random Y’s value to raise finding assembly towards same composite numbers.
 
 We exchange information and cooperate “theory group” and “programming group”, we made a program that is to perform factorization in prime numbers fast.
 
 Furthermore, as activity from this fiscal year, we decided public relations activities of FUN-ECM. As method of public relations we formed public relation (P.R) group, and we decided to make webpage can browse easily. We established targets of browsing are university students of information system.
According to P.R group, we could make webpage directed at targets half year.
% 英文キーワード
\begin{ekeyword}
Elliptic Curve Method, prime factorization, ECMNET, Twisted Edwards Curve, Extended Twisted Edwards Coordinates, RSA cryptosystem
\end{ekeyword}
\bunseki{山下哲平}
\end{eabstract}

\end{document}


=======
% 和文概要
\begin{jabstract}
 私たちのプロジェクトの目的は,より大きな桁数の素因数を見つけることである.素因数分解は,約 30年前から重要になってきている.その理由は,RSA暗号にある.RSA暗号は,安全性を2つの大きな素因数からなる合成数の素因数分解が難しいことに依存している.しかし,技術の発展とともに素因数分解が従来よりも容易になってきてしまっているため,RSA暗号が破られる可能性が高くなっている.そこで今注目されているのが楕円曲線暗号である.楕円曲線暗号は,RSA暗号と同じ鍵長で高い安全性を保障することができる.そこで私たちは素因数分解をより簡単なものとすることで,RSA暗号から楕円曲線暗号を主流とさせたい.

 私たちは,大きな素因数の発見のために,色々な文献を読んでその中から素因数分解を行うプログラムの改良法を発見する理論班と,それらの理論を利用して実際にプログラムの実装・改良を行い,プログラムを高速化させるプログラム班に分かれて活動を行った.

 理論班は,素因数分解がより高速に行われるようなアルゴリズムの発見を目標とした.楕円曲線法のプログラムは点の加算の繰り返しで行われるため,加算の計算コストを減らすことで高速な計算を可能とするための活動を行った.Atkin Morain ECPPを利用することで,従来の楕円曲線法よりも計算コストを削減できることを発見した.

 プログラム班は,前年度に作成された素因数分解プログラムをさらに高速化することを目標とした.前年度と同様に大きな数を扱うために,任意精度演算ライブラリのGMPを使用した.また,プログラムの並列実行を行うために,並列プログラムの為のAPI であるOpenMPを導入した.前年度に実装されたエドワーズ曲線よりも効率よく計算を行うため,extended twisted Edwards coordinatesを実装した.同じ合成数に対してプログラムを実行する際の因数の発見確率をあげるために,Yの値をランダムに設定した.

 また,理論班とプログラム班で情報の交換を行ったり,協力を行ったりなど,2つの班の活動により,素因数分解を高速に行うことができるプログラムが完成した.


% 和文キーワード
\begin{jkeyword}
素因数分解,楕円曲線法,ECMNET,エドワーズ曲線,射影座標,RSA暗号
\end{jkeyword}
\bunseki{山下哲平}
\end{jabstract}

%英語の概要

\begin{eabstract}

~The goal of our project team is to find prime factor as large as possible. Factorizations in prime numbers have become more important since about thirteen years ago because the difficulty of factorization in prime numbers is related to Internet security. The reason lies in the RSA. The asymmetry of RSA is based on the practical difficulty of factoring the product of two large prime numbers. However, prime factorization is getting to easier by development in technology. Therefore, RSA is less secure compared to previously. That's why Elliptic Curve Cryptography (ECC) is paid more attention than RSA now. ECC ensure safety better than RSA cryptosystem with same key length.Accordingly, we make factorization in prime numbers simplify, we would like to change main cryptosystem from RSA cryptosystem to elliptic curve cryptography.

~In order to find prime factor as large as possible, we divided two groups, one is “theory group” that is to read various literature and to find algorithm of factorizations in prime numbers to calculate faster, the other is “programming group” that is to make program to base on algorithm.

 ~“Theory group” aims to find algorithm of factorizations in prime numbers to calculate faster. The ECM program repeats process of addition law many times over, therefore we reduced calculation. To access Atkin Morain construction, we were successful in calculation faster compared to previously.

~“Programming group” aim to improve program of last year project team faster than before. To treat large number likewise last year, we used arbitrary-precision arithmetic library called GMP. Also, we parallelize the program, we introduce Open MP is API for parallel program. We implement extended twisted Edwards coordinates efficient than Edwards curve implemented last year. Also, we set random Y’s value to raise finding assembly towards same composite numbers.
 We exchange information and cooperate “theory group” and “programming group”, we made a program that is to perform factorization in prime numbers fast.
% 英文キーワード
\begin{ekeyword}
Elliptic Curve Method, prime factorization, ECMNET, Twisted Edwards Curve, Extended Twisted Edwards Coordinates, RSA cryptosystem
\end{ekeyword}
\bunseki{山下哲平}
\end{eabstract}

\tableofcontents% 目次

\mainmatter% 本文のはじまり

\chapter{背景}

ECM(楕円曲線法)を利用した素因数分解は近年重要になっており,それを利用しECM-NETにランクインすることが私たちの目的である.

\bunseki{駒ヶ嶺壮}

\section{本プロジェクトの背景}

現在インターネットを含む通信での暗号技術においての主流はRSA暗号である.RSA暗号とは公開鍵暗号の一つで,大きな合成数を素因数分解することの難しさを安全性の根拠にした暗号である.しかし,スーパーコンピューターの並行処理能力と計算能力の向上等で鍵長1024ビットのRSA暗号方式は解読される危険性が指摘されるようになった.ここで,今後の暗号技術にはRSAに変わるものとして楕円曲線暗号が使われて始めている.楕円曲線暗号は現在の暗号技術において最も重要とされている手法である.これは,ある楕円曲線における有限体上の楕円曲線の点の加算を用いることにより,RSA暗号と同じ鍵長でより解読が難しくなるからである.ここで私たちはこの楕円曲線暗号の中で核となる楕円曲線を用いた素因数分解のアルゴリズムについて考え,FUN-ECMがECM-NETにランクインを目指すことで函館から楕円曲線,素因数分解,暗号技術の重要性について発信することを目標として掲げた.

\bunseki{山下哲平}

\section{ECM-NETとは}

ECM-NETとは,楕円曲線法を用いて大きい桁数の素因数を見つけることを目的とした競争サイトである.ECM-NETには現在登録されている素因数分解よりも大きな素因数を見つけることで誰でもランクインすることが可能である.

\bunseki{駒ヶ嶺壮}

\section{課題の概要}\label{sec:gaiyou}

FUN-ECMがECM-NETへのランクインを目指すには大きい桁数の素因数を見つけなければいけないことから楕円曲線を用いた素因数分解のプログラムの並列処理と高速化を目指す.また,本プロジェクトの活動をWebサイト等を用いて外部に発信する.

\bunseki{駒ヶ嶺壮}

\chapter{到達目標}

\section{本プロジェクトにおける目的}\label{sec:mokuteki}

FUN-ECMがECM-NETにランクインするためには去年のプログラムをより改善する必要がある.この目標を達成するにあたって,2つの目標を立てることにした.

\bunseki{伊藤有輝}

\subsection{プログラムの高速化}\label{sec:goal1}

ECM-NETにランクインするためには,巨大な素因数を発見しなければならない.巨大な素因数を発見するためには桁数の大きい合成数を素因数分解する必要があるが,それには多大な時間がかかってしまう.また,ECMは1度の試行で素因数を必ず発見できるとは限らず,複数回の試行が必要となる.そのためプログラムの処理を効率の良いアルゴリズムに変更し,処理を高速化させる必要がある.この目標を達成するにあたって,2つの目標を立てることした.

\begin{itemize}
\item 昨年度のプログラムのアルゴリズムの理解
\item 昨年度のプログラムの書き換えたものの実装
\end{itemize}

まず,昨年度のプログラムを高速化するにはアルゴリズムの理解が必要である.また,楕円曲線法では大学までの学習で使用していない数学の概念を使用するため,基礎学習を行う.

\bunseki{伊藤有輝}

\subsection{FUN-ECMの活動発信}\label{sec:goal2}

今年度では,ただランクインを目指すだけでなく,函館から楕円曲線,素因数分解の重要性について発信することに決め,ホームページを設立することとした.

\bunseki{伊藤有輝}

\section{課題達成の為の班分け}
前年度のプロジェクトでは前期で楕円曲線法についての学習を行い,後期でアルゴリズムの提案・実装を行っていた.しかし,このような日程でプロジェクトを進行していくと以下のような問題が発生した.

\begin{itemize}
\item 実際にプログラムを実装する期間が少ない
\item 完成したプログラムを試行する期間が少ない
\item 巨大な合成数の分解を行いにくい
\end{itemize}

そのため,本プロジェクトでは5月中旬まで全員で最低限の基礎学習を行い,そこから理論班とプログラミング班の2つに分けて作業を行うこととした.また,後期には広報班を作成し,3つの作業を並行で行うこととした.以下にそれぞれの班の課題について述べる.

\begin{description}
 \item[理論班]\mbox{}\\ 
	    ECMについて理解を深め,高速化の新たなアルゴリズムを提案する.
 \item[プログラミング班]\mbox{}\\
	    基礎学習や理論班がまとめたアルゴリズムを実装し高速化を行う.
 \item[広報班]\mbox{}\\
	    ECMについて理解してもらえるようなWebページの作成をする.
\end{description}
\bunseki{伊藤有輝}

\chapter{前期活動内容}

プロジェクトが始まった当初,ほぼ全員楕円曲線についての前提知識がなかったため,昨年も前提知識を身に着けるために使われた全員楕円曲線についての資料を全員で輪読し,理解した.その際,理解できなかったところを由良先生,白勢先生に解説してもらった.それにより,楕円曲線法のアルゴリズムを理解するためにあたっての基礎知識を学んだ.その後,プロジェクト全体をプログラムの高速化につながる理論を探し,学習してアルゴリズムをノートにまとめる理論班,理論班がノートにまとめたアルゴリズムをプログラムに実装するプログラミング班に分けてプロジェクトを進めた.

\bunseki{伊藤有輝}

\section{基礎学習}

去年のプログラムを理解するために5月の中旬まではメンバ全員が教授の指導のもとで楕円曲線法のアルゴリズムや基礎知識ついての基礎学習を行った.具体的な内容は以下の通りである.
\begin{description}
 \item[有限体]\mbox{}\\ 
	      $素数pに対し,0からp-1までの整数の集合\mathbb{F}_p=\{0,1,…,p-1\}を有限体と言う.\mathbb{F}_pでは四則演算が可能であり,ECMではこの範囲で考える.$
 \item[Euclidの互除法]\mbox{}\\
		$自然数a,b(a≧b)に対して以下の操作を繰り返し余りが0になるまで行うことによってa,bの最小公倍数を求めるものである.$

\begin{algorithm}[h]                   
\caption{Euclidean Algorithm}
\label{alg E}                          
\begin{algorithmic}                  
\REQUIRE $a,b \in \mathbb{N} , \quad a,b \neq 0,\quad a\ge b$
\ENSURE $\gcd (a,b)$
\WHILE {$b \neq 0$}
\STATE $q \leftarrow a/b$
\STATE $r \leftarrow a\mod b$
\STATE $a \leftarrow b$
\STATE $b \leftarrow r$
\ENDWHILE
\end{algorithmic}
\end{algorithm}
$a,bの最大公約数を\gcd (a,b)と表記できる.$
	
 \item[拡張Euclidの互除法]\mbox{}\\
	$与えられた整数a,b,cに対し,未知数x,yに関する一次方程式ax+by=c$の整数解を求める問題を一次不定方程式という.ここで,$自然数a,bに関する一次不定方程式ax+by=gcd(a,b)を満たす無数の整数x,yは拡張$Euclidの互除法を用いることで効率よく求めることができる.これはEuclidの互除法で行った操作を逆に行うことで解を得る.$gcd(174,69)=3を例にとって考える.$

	\begin{align*} 
		174/69&=2*69+36 \\
		69/36&=1*36+33 \\
		36/33&=1*33+3 \\
		33/3&=11 	
 	\end{align*}
	となるので
	\begin{align*} 
	3&=36-33*1 \\
	&=36-(69-36*1)*1 \\
	&=69*(-1)+36*2 \\
	&=69*(-1)+(174-69*2)*2 \\
	&=174*2+69*(-5)
 	\end{align*}

以上より,$174x+69y=3の解(x,y)=(2,-5)$を得ることができる。有限体$\mathbb{F}_p$において除算$a/b$を計算する場合, $pとb$は互いに素なので, 拡張Euclidの互除法により不定方程式$px + by = 1 の解(x, y)$を求めることができる。このとき$px+ by= 1$となるので, 有限体$\mathbb{F}_p$上では$by = 1となり,両辺をbで割ることで,b^−1 = y$が成立する。したがって$a ÷ b = a×b-1=a × y$と変形することで, 除算を乗算に置き換えて計算できる。プログラムにおいて、除算を乗算に置き換えることは計算量の削減につながるが、今回のプロジェクトではGMPライブラリを用いたことでこれを実装することはなかった。


	
\item[楕円曲線の定義方程式]\mbox{}\\
	$a,b \in \mathbb{F}_pに対してy^2 = x^3 + ax + bで定義される曲線を素体Fp上の楕円曲線という.$

\item[楕円曲線の加算・2倍算]\mbox{}\\
	$(加算) 楕円曲線上のある2点P,Qを通る直線をℓとすると,楕円曲線と直線 \ell の3つ目の交点R’(=P×Q)のx軸に関する対称点をRとする。このとき2点P,Qの和をR=P+Qと定義し、楕円曲線の加算という。$
	
$(2倍算) 楕円曲線上の1点Pで加算を考えるときは2点P,Pの通る直線(=Pの接線)をℓとして考える。この時、楕円曲線と直線 \ell のP以外の交点のx軸に関する対称点をRとしたとき、R=P+P=2Pとできる。これが楕円曲線の2倍算である。
$
	
\item[楕円曲線のスカラー倍]\mbox{}\\
	$点Pと整数mを使用して,mP=P+P+P+P+・・・・+P(m個の和)と表すことができる.これを楕円曲線のスカラー倍という.$

\item[楕円曲線法のアルゴリズム ]\mbox{}\\
	$Nを素因数分解したい合成数とする.\mathbb{Z}/N\mathbb{Z}上で,楕円曲線Eを構成して,点$
	\begin{equation}
	P \in E(\mathbb{Z}/N\mathbb{Z})
	\end{equation}
	$をとる.初めにPの座標を決めてからEを構成しても良い.$
	
	$次に適切なB_1,L=2,3,・・・B_1の最小公倍数とする.LPの計算の過程で生じる点の座標の分母dが\gcd(N,d) \neq 1となるとNの約数を発見できる.$
	
	$最期まで\gcd (N,d)=1ならば,EとPを選びなおしてやり直す.適切なB_1を選ぶことで,ECMは高速な素因数分解法になることが知られている.$
\end{description}
以上のことを基礎学習として学んだ.以下の章ではに2班に分かれた後の理論班の活動内容を記述する.

\bunseki{橋本和典}

\section{理論班}
理論班では新たなアルゴリズムを探し,プログラミング班に新たな高速化手法の提案を行った.以下に具体的な内容を述べる.

\subsection{Twisted Edwards Curveの理解}
ECMの高速化アルゴリズムを実装するにあたって,先人の知恵を得ようと思いインターネットで類似研究の論文を検索し,その論文を解読することによって高速化アルゴリズムをプログラムに導入しようと考えた.その際,Twisted Edwards Curves Revisitedというエドワーズ曲線についての英語の論文が見つかったため,私たちはこの論文を読解することにした.

この論文は,最初に一般的な楕円曲線アルゴリズムより,エドワーズ曲線の方が計算コストは低く,速いスピードで素因数を求めることができるということが説明されており,そのエドワーズ曲線の数学的な理論とプログラム実装のためのアルゴリズムが書かれていた.

エドワーズ曲線については基礎学習で学んでいなかったため,私たちはエドワーズ座標を学習した.その中では射影座標が使用されていた.射影座標とは一般的なの座標(x,y)に対して$x=\frac{X}{Z},y=\frac{Y}{Z}を満たすX,Y,Zを用いて(X,Y,Z)と表す座標であり,射影座標を用いると$ECMアルゴリズムを高速化することができる.具体的な定義は以下の通りである.

\begin{itembox}[l]{射影座標}
\begin{center}
$(X,Y,Z)=(\lambda X, \lambda Y, \lambda Z)=$$(\displaystyle \frac{X}{Z}$,$\displaystyle \frac{Y}{Z}$,1) $(Z\neq0)$
\end{center}
\end{itembox}

理論班では,この拡張エドワーズ座標の理論を学ぼうとしたが,知識が乏しく,わからない変数が出てきたため,アルゴリズムだけを理解し,定義,証明などの理論を理解することはあきらめた.

\bunseki{伊藤有輝}

\subsection{Atkin-Morain ECPP}
次にAtkin-Morain ECPPというECMの初期座標を決定するアルゴリズムの理解に励んだ.昨年度まではECMの初期座標として(2,2)を用いて,素因数分解が完了できなければ(2,3),(2,4)…といったようにY座標を1ずつ動かすようにしていたが,今年度では少しでもを因数を見つける確率を上げることが見込めるAtkin-Morain ECPPを理解することにした.Atkin-Morain ECPPでは新たな楕円曲線$T^2=S^3-8S-32の点を用意し,(S,T)=(12,40)に対してn(S,T)の座標(s,t)$を用いて以下を定める.

\begin{center}
\begin{equation}
\alpha =\cfrac{(s-9)+1}{t+25}  ,  \beta = \cfrac{2\alpha (4\alpha +1)}{8\alpha^2-1}
\end{equation}
\end{center}
これらを用いることによって、素因数分解に用いる楕円曲線の初期座標を求めることができる。具体的には以下の通りである。

\begin{algorithm}[h]                   
\caption{Atkin-Morain ECPP Algorithm}
\label{alg ECPP}                          
\begin{algorithmic}                  
\REQUIRE $\alpha,\beta,s,t,\in \mathbb{N}$
\ENSURE $(X,Y)$
\STATE $(s,t) \leftarrow(12,40)$
\WHILE {Prime factor is not found}
\STATE $\alpha \leftarrow \cfrac{(s-9)+1}{t+25}$
\STATE $\beta \leftarrow \cfrac{2\alpha (4\alpha +1)}{8\alpha^2-1}$
\STATE $d \leftarrow \cfrac{2(2\beta -1)^2-1}{(2\beta -1)^4}$
\STATE $E:x^2+y^2=1+dx^2y^2$
\STATE $X \leftarrow \cfrac{(2\beta -1)(4\beta -3)}{6\beta -4}$
\STATE $Y \leftarrow \cfrac{(2\beta-1)(t^2+50t-2s^3+27s^2-104)}{(t+3s-2)(t+s+16)}$
\STATE Run ECM with $E:x^2+y^2=1+dx^2y^2$ and $(X,Y)$
\STATE $(s,t) \leftarrow 2(s,t)$
\ENDWHILE
\end{algorithmic}
\end{algorithm}

このアルゴリズムを用いると具体的には従来の1.5倍ほど高速化できる見込みであるが、これは論文上のデータである。したがって、後期はプログラム班が実装し、どのくらい速くなるかどうかを検証したいと考えている。

\bunseki{伊藤有輝}

\section{プログラミング班}
プログラミング班では,昨年度のFUN-ECMプロジェクトで作成したECMプログラムをさらに高速化するために,4月から5月にかけて行った全体での基礎学習や,理論班がまとめた理論・アルゴリズムを元にプログラムを変更した.主に,射影座標やextended twisted Edwards coordinatesを用いて乗算・除算を減らすことによって高速化を図った.また,前年度のプログラムの不具合等も改善した.具体的には以下の通りである.

\bunseki{源啓多}

\subsection{座標変換の際の冗長なコストの削減}\label{sec:alg1}
前年度のプロジェクトで作成されたECMプログラムでは,スカラー倍をする際の座標をアフィン座標から射影座標に変換することで計算効率を上昇させていた.このアフィン座標から射影座標への変換は複数回呼び出される為,ECMプログラムの計算コストに影響する.Algorithm \ref{alg:algP}にアルゴリズムを記す.

\begin{algorithm}[H]
\caption{Affine Coordinates to Projective Coordinates (Past ver.)}
\label{alg:algP}                          
\begin{algorithmic}                  
\REQUIRE $(AX,AY)$ is Affine, $(PX,PY,PZ)$ is Projective, $N \ge 2 $
\ENSURE $(PX,PY,PZ)$
\STATE $Z \leftarrow Random(0 \le Z < N)$
\IF {$Z=0$}
\STATE $Z \leftarrow 1$
\ENDIF
\STATE $AX \leftarrow AX \times Z$
\STATE $AY \leftarrow AY \times Z$
\STATE $AX \leftarrow AX \mod N$
\STATE $AY \leftarrow AY \mod N$
\STATE $(PX,PY,PZ) \Leftarrow (AX,AY,Z)$
\end{algorithmic}
\end{algorithm}


前述の冗長部分として乗算が2回と$mod$の計算が2回発生している.プログラミング班では,$Z$の値を1に設定することで乗算と$mod$の計算を省略できると考えた.プログラムを一通り読み直し,問題が発生しないことを確認したのち,新たなアルゴリズムを実装した.Algorithm \ref{alg:algN}に新しいアルゴリズムを示す.

\begin{algorithm}[H]                   
\caption{Affine Coordinates to Projective Coordinates (New ver.)}
\label{alg:algN}                          
\begin{algorithmic}                  
\REQUIRE $(AX,AY)$ is Affine, $(PX,PY,PZ)$ is Projective, $N \le 2 $
\ENSURE $(PX,PY,PZ)$
\STATE $Z \leftarrow 1$
\STATE $(PX,PY,PZ) \Leftarrow (AX,AY,Z)$
\end{algorithmic}
\end{algorithm}

\bunseki{源啓多}

\subsection{Extended twisted Edwards coordinatesの実装}\label{sec:alg2}
前年度のプロジェクトで作成されたECMプログラムでは,twisted Edwards curveを利用している.今回のプログラミング班ではさらにextended twisted Edwards coordinatesを用いた.extended twisted Edwards coordiantesはエドワーズ曲線のスカラー倍を高速化するための座標系であり,以下で定義される補助座標Tを加えた4つの座標でスカラー倍を行う.

\begin{itembox}[l]{Extended twisted Edwards coordinates}
射影座標(X,Y,Z)をに対し,T=$\cfrac{XY}{Z}$という補助座標を加える.これをExtended twisted Edwards coordinatesと呼ぶ.
\begin{center}
$(X,Y,Z) \rightarrow (X,Y,T,Z)$
\end{center}
\end{itembox}
\bunseki{源啓多}

\subsection{楕円曲線の生成法の変更}\label{sec:alg3}
楕円曲線法を利用したECMプログラムは,楕円曲線を生成しその座標を利用し素因数分解を行うプログラムである.また,本プロジェクトで素因数分解しようと試みている合成数は200桁前後のため,1度の試行では素因数分解できないことが多くある.よって,同じ合成数に対して複数回の試行をすることを想定してプログラムを作成する必要がある.前年度のプログラムでは,楕円曲線を生成する際に,Y値をfor文のカウンタを利用して1から順に決めるアルゴリズムを採用していた.そのため,複数回試行した際に同じ曲線を使用してしまうことが多くあり,効率が落ちていたと仮定した.そこで曲線を生成する際に使用しているY値に乱数を使用することとした.

\bunseki{源啓多}

\section{中間発表}

\subsection{準備}
\begin{description}
\item[ポスター]\mbox{}\\
初めに,前年度のプロジェクトで作成されたポスターを参考に構成を決定した.次に,概要,基礎学習,理論班,プログラミング班の4つの項目に分け,作成を分担した.ポスターの作成には「Microsoft PowerPoint」というソフトウェアを使用した.ポスターが完成次第,理論班・プログラミング班でレビューを行い,誤字脱字等を修正した.しかしポスターレビューが不十分だったため,最終的に完成したポスターで誤植が見つかってしまった.

\bunseki{亀谷浩也}

\item[プレゼンテーション資料]\mbox{}\\
本プロジェクトの内容を説明するには,ポスターだけでは足りないと判断しプレゼンテーション資料を作成することに決定した.作成にあたって,まず1名がプレゼンテーションの大まかな流れを作成し,各自作成する章を分担した.プレゼンテーション資料の作成には「Microsoft PowerPoint」というソフトウェアを使用した.また,一度完成したプレゼンテーション資料を先生にレビューしていただき,資料中のグラフの不備や内容についての助言を受けた.それを受け,文章や図の修正を行った.これにより,より見やすいプレゼンテーション資料が完成した.

\bunseki{亀谷浩也}

\item[原稿]\mbox{}\\
前述のプレゼンテーション資料の作成と並行して,発表用の原稿の作成を行った.こちらも1名が大まかな流れを作成し,各自作成する章を分担した.特に楕円曲線法については,何も知らない聴衆でもわかりやすく説明できるように,専門的な用語を最小限にするように注意して作成した.何度か原稿とプレゼンテーション資料を使用しプレゼン練習を行い,伝わりにくい表現や冗長な表現を修正した.
\end{description}

\bunseki{亀谷浩也}

\subsection{発表}
発表は前後半で4人ずつに分かれ,発表を行った.それぞれが自分の担当する部分を読み上げ,その間他の3人は評価アンケート配布や,ポスターに関しての質問に対応した.発表途中にプロジェクターの電源が落ちてしまうというアクシデントがあったが,落ちている間はPCの画面を直接見せることでプレゼンを行い,他の3人で復旧作業を行った.発表後に評価アンケートの集計を行った結果、発表技術は10点中平均7.1点、発表内容は10点中7.5点だった。コメントでは内容を理解していた人と全く理解できない人が分かれていたため、さらに前提知識のない聴衆にも伝わるような内容にしていきたい。

\bunseki{亀谷浩也}

\chapter{後期活動内容}

\bunseki{亀谷浩也}

\section{理論班}

\bunseki{亀谷浩也}

\subsection{プログラムの検証}

\bunseki{亀谷浩也}

\section{プログラミング班}

\bunseki{亀谷浩也}

\subsection{Atkin-Morain ECPPの実装}

\bunseki{亀谷浩也}

\subsection{新しい改良法の調査}

\bunseki{亀谷浩也}

\subsubsection*{疑似的2次拡大環状での楕円曲線法の提案}

\bunseki{亀谷浩也}

\subsubsection*{復習種類の曲線を組み合わせた楕円曲線法の提案}

\bunseki{亀谷浩也}

\subsubsection*{Stage2の提案}

\bunseki{亀谷浩也}

\subsection{Stage2}

\bunseki{亀谷浩也}

\subsubsection*{基本的なStage2の実装}

\begin{algorithm}[H]                   
\caption{Basic ECM Algorithm}
\label{alg:B}                          
\begin{algorithmic}                  
\REQUIRE $N$ is composite number, $E$ is elliptic curve, $P = (x_0, y_0, Z_0) \in E(Z_n)$ is initial point, $B_1$ is smoothness bound for Phase 1, $B_2$ is smoothness bound for Phase 2, $B2 \ge B1.$
\ENSURE $q$ is factor of $N$, $1 \le q \leq N$, or FAIL.
\STATE \bfseries{Phase 1.}
\STATE $k \gets \prod_{p \leq B_1} p^{\log{p} B_1}$
\STATE $Q_0 \gets kP_0$
\STATE $q \gets \gcd(z_{Q_0},N)$
\IF {$q \ge 1$}
\STATE return $q$
\ELSE
\STATE go to Phase 2
\ENDIF
\STATE \bfseries{Phase 2.}
\STATE $d \gets 1$
\FOR {each prime $p = B_1$ to $B_2$} 
\STATE $(x_{pQ_0},y_{pQ_0},z_{pQ_0}) \gets pQ_0$
\STATE $d \gets d*Z_{pQ_0} (mod N)$
\ENDFOR
\STATE $q \gets \gcd(d,N)$
\IF {$q \ge 1$}
\STATE return $q$
\ELSE
\STATE return FAIL
\ENDIF
\end{algorithmic}
\end{algorithm}

\bunseki{亀谷浩也}

\subsubsection*{Montgomely ladderを利用したStage2の提案}

\bunseki{亀谷浩也}

\section{広報班}

\bunseki{亀谷浩也}

\subsection{ウェブページの作成}

\bunseki{亀谷浩也}

\chapter{プロジェクト内のインターワーキング}
\begin{itemize}
\item 池野竜將(プロジェクトリーダー・プログラミング班)
 \begin{enumerate}
 \renewcommand{\labelenumi}{(\arabic{enumi})}
 \item 楕円曲線法の基礎を学んだ.
 \item 大まかな作業スケジュールを作成し,進捗管理を行った.
 \item 源と協力して前年度のECMプログラムを理解した.
 \item 源のコーディング作業にアドバイスをした.
 \item 理論班からのプログラミング班に関しての質問に回答し,必要があれば聞かれた内容を源に伝えた.
 \item 中間発表会に向けて,プレゼンテーション資料・原稿の原案を作成した.
 \item 中間発表会に向けて,「プログラミング班」の部分のプレゼンテーション資料を作成した.
 \end{enumerate}
 
\item 源啓多(プログラミング班)
 \begin{enumerate}
 \renewcommand{\labelenumi}{(\arabic{enumi})}
 \item 楕円曲線法の基礎を学んだ.
 \item 池野と協力して前年度のECMプログラムを理解した.
 \item ECMプログラムのバージョン管理の為,Gitを学んだ.
 \item 前年度のECMプログラムの実装上のミス(\ref{sec:alg1})を改善した.
 \item ECMプログラム改善のために,新たなアルゴリズム(\ref{sec:alg2}, \ref{sec:alg3})の実装を行った.
 \item 中間発表会に向けて,プログラミング班のプレゼンテーション資料・原稿を作成した.
 \item Stage2の解読・実装をいち早く進めた.
 \item 解析班の作業を助けるためのマクロを作成した.
 \item 広報班に協力し,ウェブページの作成の手助けをした.
 \end{enumerate}
 
\item 山下哲平(理論班)
 \begin{enumerate}
 \renewcommand{\labelenumi}{(\arabic{enumi})}
 \item 楕円曲線法の基礎を学んだ.
 \item 伊藤・駒ヶ嶺と協力してEdwards Curveを利用したECMアルゴリズムの読解を行い,プログラミング班に提案を行った.
 \item 伊藤・駒ヶ嶺と協力してAtkin-Morain ECPPアルゴリズムの理解に取り組んだ.
 \item 中間発表会に向けて,「背景」の部分についてポスターをを作成した.
 \item プログラミング班の要請でプログラムの速度について簡易的な検証を行った.
 \item 伊藤と協力してウェブページの基本的な要素を作成した.
 \item 最終報告書の広報班ページを作成した.
 \end{enumerate}
 
\item 伊藤有輝(理論班)
 \begin{enumerate}
 \renewcommand{\labelenumi}{(\arabic{enumi})}
 \item 楕円曲線法の基礎を学んだ.
 \item 駒ヶ嶺と協力して,エドワーズ曲線の式が導き出される過程を学んだ.
 \item 駒ヶ嶺・山下と協力し,Edwards Curveを利用したECMアルゴリズムの読解を行った.
 \item 駒ヶ嶺・山下と協力し,Atkin-Morain ECPPアルゴリズムの理解に取り組み,プログラミング班に提案を行った.
 \item 中間発表会に向けて,「理論班」の部分のプレゼンテーション資料を作成した.
 \item 源・池野と協力し,Githubの使い方を理解して広報班に伝えた.
 \item 山下と協力してウェブページの基本的な要素を作成した.
 \end{enumerate}
 
\item 駒ヶ嶺壮(理論班)
 \begin{enumerate}
 \renewcommand{\labelenumi}{(\arabic{enumi})}
 \item 楕円曲線法の基礎を学んだ.
 \item 伊藤と協力して,エドワーズ曲線の式が導き出される過程を学んだ.
 \item 山下・伊藤と協力してEdwards Curveを利用したECMアルゴリズムの読解を行った.
 \item 山下・伊藤と協力してAtkin-Morain ECPPアルゴリズムの理解に取り組んだ.
 \item 中間発表会に向けて,「理論班」の部分のポスターを作成した.
 \item 広報班のウェブページ作成のため,過去の作業ログを見直し,まとめた.
 \end{enumerate}
 
\item 橋本和典(理論班)
 \begin{enumerate}
 \renewcommand{\labelenumi}{(\arabic{enumi})}
 \item 楕円曲線法の基礎を学んだ.
 \item 千葉・亀谷と協力して入門書を読み,基礎学習を行った.
 \item 亀谷と協力して基礎学習を簡潔にまとめた解説ノートを作成した.
 \item 中間発表会に向けて,千葉・亀谷と協力して来るであろう質問を予測して対策を行った.
 \item ECMの改善に直結するような文献を探した。
 \item 中間発表会に向けて、ポスターの「理論班」の章を英訳した。
 \item 理論班で検証を行う際に,管理者として中心となって作業した.
 \item 行った検証の結果をまとめ,グラフ化して見やすくした.
 \end{enumerate}
 
\item 千葉大樹(理論班)
 \begin{enumerate}
 \renewcommand{\labelenumi}{(\arabic{enumi})}
 \item 楕円曲線法の基礎を学んだ.
 \item 亀谷・橋本と協力して入門書を読み,基礎学習を行った.
 \item 中間発表会に向けて,ECMについての英論文から重要な単語を抜粋し解説した.
 \item 中間発表会に向けて,亀谷・橋本と協力して来るであろう質問を予測して対策を行った.
 \item 中間発表会に向けて、ポスターの「プログラミング班」の章を英訳した。
 \item 理論班で検証を行う際に,実際にプログラムを動かし,データを全体に共有した.
 \end{enumerate}
 
\item 亀谷浩也(理論班)
 \begin{enumerate}
 \renewcommand{\labelenumi}{(\arabic{enumi})}
 \item 楕円曲線法の基礎を学んだ.
 \item 橋本・千葉と協力して入門書を読み,基礎学習を行った.
 \item 橋本と協力して,基礎学習を簡潔にまとめた解説ノートを作成した.
 \item 中間発表会に向けて,橋本・千葉と協力して来るであろう質問を予測して対策を行った.
 \item 中間発表会に向けて、ポスターの「概要・基礎学習」の章を英訳した。
 \item 理論班で検証を行う際に,データの管理やまとめを手伝い,橋本の補佐として活動した.
 \end{enumerate}
 
\end{itemize}

\bunseki{池野竜將}

\chapter{前期活動成果}


本プロジェクトでは,理論班で理解することに成功した高速化アルゴリズムをプログラミング班に伝え,プログラミング班がそのアルゴリズムを実装することによりECMプログラムを作成した.

\bunseki{千葉大樹}

\section{理論班}

理論班は,活動内容で示した,エドワーズ曲線においての射影座標を用いたスカラー倍楕円曲線プログラムでの変数の点の与え方のアルゴリズムの改善点を発見した.乗算の回数,除算の回数が減少したことにより素因数を発見する効率が理論上1.5倍減少したが,実装前との計算コストの実数値の比較についてはまだできていない.また,Atkin-Morain ECPPのアルゴリズムの理解をすることに成功した.これを実装することにより,ECMによって素因数pが見つかる確率は,位数があらかじめ小さな因数dを持つ曲線のみを使用した場合,ランダムに動く部分のサイズがpからp=dに減少するため因数分解に成功する確率を高めることができる.しかし,Atkin-Morain ECPPの理論については理解することができなかった.そのため,プログラミング班にはAtkin-Morain ECPPの実装のためのアルゴリズムを書き起こしレポート用紙を渡すことにより,ECM USING EDWARDS CURVEの読解を終了した.

\bunseki{駒ヶ嶺壮}

\section{プログラミング班}
プログラミング班では,新たなアルゴリズムを実装し,理論上は\ref{tab:cost}のように計算量が減少することが分かった.詳細な実験は行っておらず有意な差があるかどうかは確認できていない.だが,実際に素因数分解を行った結果,処理が早くなっていることが確認できた.

\begin{table}
\begin{center}
\caption{昨年度と今年度のプログラムの計算コストの比較}\label{tab:cost}
\begin{tabular}{ccc}
\hline
& 2倍算 & 2倍算→加算\\
\hline
昨年度 & 3{\bf M}+4{\bf S}+1{\bf D}\footnotemark & 13{\bf M}+5{\bf S}+3{\bf D}\\
今年度 & 3{\bf M}+4{\bf S}+1{\bf D} & 12{\bf M}+4{\bf S}+1{\bf D}\\
\hline
\end{tabular}
\end{center}
\end{table}
\footnotetext{{\bf M}:乗算,{\bf S}:2乗算,{\bf D}:楕円曲線の係数a,dを用いた乗算}

また,実際に巨大な合成数を分解し,昨年度のプログラムとの性能を比較することにした.評価するにあたって,2015年度に作成されたプログラムでテストに使用されていた合成数$10^{306}+1$を素因数分解することで,以前のプログラムとの比較をすることとした.2015年度のプログラムでこの合成数を分解した結果,発見されたもっとも大きな素因数は157538980319816607(21桁)であった.同様に今年改善されたプログラムで分解した結果,発見されたもっとも大きな素因数は112544281755782732673671367061(30桁)であり,より大きな素因数を見つけることができるように改善された.

\bunseki{源啓多}

\chapter{後期活動成果}

\section{理論班}

\bunseki{橋本和典}

\section{プログラム班}

\bunseki{橋本和典}

\section{広報班}

\bunseki{橋本和典}

\chapter{まとめ}

\section{前期活動結果}

前期は参考資料,論文,担当教員の白勢先生の講義による楕円曲線法の理解から始め,楕円曲線が楕円曲線法においていつどのように使われるかを理解した.その後,理論班,プロジェクト班の2班に分かれ作業を行った.理論班は,論文,入門書の読解をし,プログラム高速化のための改善案を出すことに成功した.しかし,前期中にプログラミング班が実装することはできなかった.プログラミング班は前年度のプロジェクトで作成されたECMプログラムを理解した.その後,実装ミスの改善や,新たなアルゴリズムの実装を行い,計算コストの減少に成功した.

\bunseki{千葉大樹}

\section{後期の展望}

後期は,理論班が作成したAtkin-Morain ECPPアルゴリズムを実装し,さらにECMプログラムの改善を図る.また,大きな合成数の分解を続けECMNETへのランクインを目指す.加えて,前期中に活動できなかった広報について新たに班を設置し活動していく.

\bunseki{橋本和典}

\section{後期活動結果}

\bunseki{橋本和典}

\section{全体を通して}

\bunseki{橋本和典}

\appendix
\chapter{新規習得技術}

\begin{itemize}
\item PARI/GPの使用
\item Microsoft PowerPointの使用
\item Gitの使用
\item GitHubの使用
\item Xeno Phiの使用
\item functionviewの使用
\end{itemize}

\bunseki{橋本和典}

\chapter{相互評価}

\bunseki{橋本和典}



%\backmatter
>>>>>>> origin/master

% 目次
\tableofcontents

% 本文のはじまり
\mainmatter

%背景
% 両面印刷する場合は `openany' を削除する
\documentclass[openany,11pt,papersize]{jsbook}

%パッケージの読み込みなど
\input{settings.tex}

\begin{document}

\chapter{背景}

ECM(楕円曲線法)を利用した素因数分解は,実査のインターネットで使われる暗号技術の安全性の確認に必須であるため近年重要になってきており,それを利用しECM-NETにランクインすることが私たちの目的である.

\bunseki{駒ヶ嶺壮}

\section{本プロジェクトの背景}

現在インターネットを含む通信での暗号技術においての主流はRSA暗号である.RSA暗号とは公開鍵暗号の一つで,大きな合成数を素因数分解することの難しさを安全性の根拠にした暗号である.しかし,スーパーコンピューターの並列処理能力と計算能力の向上等でRSA暗号方式は近い将来解読される危険性が指摘されるようになった.ここで,今後の暗号技術にはRSAに代わるものとして楕円曲線暗号が注目され始めている.楕円曲線暗号は現在の暗号技術において最も重要とされている手法である.これは,暗号化・復号においてある楕円曲線における有限体上の楕円曲線の点の加算を用いることにより,RSA暗号と同じか議長でより解読が難しくなるからである.ここで私たちはこの楕円曲線暗号の中で核となる楕円曲線を用いた素因数分解のアルゴリズムについて考え,FUN-ECMがECM-NETにランクインを目指すことで函館から楕円曲線,素因数分解,暗号技術の重要性について発信することを目標として掲げた.

\bunseki{山下哲平}

\section{ECM-NETとは}

ECM-NETとは,楕円曲線法を用いて大きい桁数の素因数分解をみつけることを目的とした競争サイトである.ECM-NETには現在登録されている素因数分解よりも大きな素因数を見つけることで誰でもランクインすることが可能である.過去にランクインした日本人はK.Aoki氏とT.Izu氏の2名である.

\bunseki{駒ヶ嶺壮}

\section{課題の概要}\label{sec:gaiyou}

FUN-ECMがECM-NETへのランクインを目指すには大きい桁数の素因数を見つけなければいけないことから楕円曲線を用いた素因数分解のプログラムの並列処理と高速化を目指す.また,本プロジェクトの活動をWebサイト等を用いて外部に発信する.

\bunseki{駒ヶ嶺壮}

\end{document}


%到達目標
% 両面印刷する場合は `openany' を削除する
\documentclass[openany,11pt,papersize]{jsbook}

%パッケージの読み込みなど
% 報告書提出用スタイルファイル
\usepackage[final]{funpro}%最終報告書
%\usepackage[middle]{funpro}%中間報告書

% 画像ファイル (EPS, EPDF, PNG) を読み込むために
\usepackage[dvipdfmx]{graphicx,color}

%数式の表示に利用するため
\usepackage{amsmath,amssymb}

%アルゴリズムの表示に利用するパッケージ
\usepackage{algorithm}
\usepackage{algorithmic}

%枠をつけるためのパッケージ
\usepackage{ascmac}

%図の位置調整パッケージ
\usepackage{here}

%付録を作成するためのパッケージ
\usepackage{appendix}

%ドキュメント管理用パッケージ
\usepackage{docmute}


% ここから -->
\usepackage{calc,ifthen}
\newcounter{hoge}
\newcommand{\fake}[1]{\whiledo{\thehoge<70}{#1\stepcounter{hoge}}%
  \setcounter{hoge}{0}}
% <-- ここまで 削除してもよい


% 年度の指定
\thisYear{2016}

% プロジェクト名
\jProjectName{FUN-ECM プロジェクト}

% [簡易版のプロジェクト名]{正式なプロジェクト名}
% 欧文のプロジェクト名が極端に長い(2行を超える)場合は,短い記述を
% 任意引数として渡す.
%\eProjectName[Making Delicious curry]{How to make delicious curry of Hakodate}
\eProjectName{FUN-ECM Project}


% <プロジェクト番号>-<グループ名>
\ProjectNumber{15-A}

% グループ名
\jGroupName{Aグループ}
\eGroupName{A Group}

% プロジェクトリーダ
\ProjectLeader{1014129}{池野竜將}{Ryusuke Ikeno}

% グループリーダ
\GroupLeader  {1014129}{池野竜將}{Ryusuke Ikeno}

% メンバー数
\SumOfMembers{8}
% グループメンバ
\GroupMember  {1}{1014068}{駒ヶ嶺壮}{Sou Komagamine}
\GroupMember  {2}{1014109}{伊藤有輝}{Yuki Ito}
\GroupMember  {3}{1014129}{池野竜將}{Ryusuke Ikeno}
\GroupMember  {4}{1014137}{千葉大樹}{Daiju Chiba}
\GroupMember  {5}{1014164}{橋本和典}{Kazunori Hashimoto}
\GroupMember  {6}{1014168}{山下哲平}{Teppei Yamashita}
\GroupMember  {7}{1014209}{源啓多}{Keita Minamoto}
\GroupMember  {8}{1013150}{亀谷浩也}{Hiroya Kametani}

% 指導教員
\jadvisor{白勢政明,由良文孝}
% 複数人数いる場合はカンマ(,)で区切る.カンマの前後に空白は入れない.
\eadvisor{Masaaki Shirase, Fumitaka Yura}

% 論文提出日
\jdate{2016年7月27日}
\edate{July~27, 2016}


\begin{document}

\chapter{到達目標}

\section{本プロジェクトにおける目的}\label{sec:mokuteki}

FUN-ECMがECM-NETにランクインするためには去年のプログラムをより改善する必要がある.この目標を達成するにあたって,2つの目標を立てることにした.

\bunseki{伊藤有輝}

\subsection{プログラムの高速化}\label{sec:goal1}

ECM-NETにランクインするためには,巨大な素因数を発見しなければならない.巨大な素因数を発見するためには桁数の大きい合成数を素因数分解する必要があるが,それには多大な時間がかかってしまう.また,ECMは1度の試行で素因数を必ず発見できるとは限らず,複数回の試行が必要となる.そのためプログラムの処理を効率の良いアルゴリズムに変更し,処理を高速化させる必要がある.この目標を達成するにあたって,2つの目標を立てることした.

\begin{itemize}
\item 昨年度のプログラムのアルゴリズムの理解
\item 昨年度のプログラムの書き換えたものの実装
\end{itemize}

まず,昨年度のプログラムを高速化するにはアルゴリズムの理解が必要である.また,楕円曲線法では大学までの学習で使用していない数学の概念を使用するため,基礎学習を行う.

\bunseki{伊藤有輝}

\subsection{FUN-ECMの活動発信}\label{sec:goal2}

今年度では,ただランクインを目指すだけでなく,函館から楕円曲線,素因数分解の重要性について発信することに決め,ホームページを設立することとした.

\bunseki{伊藤有輝}

\section{課題達成の為の班分け}
前年度のプロジェクトでは前期で楕円曲線法についての学習を行い,後期でアルゴリズムの提案・実装を行っていた.しかし,このような日程でプロジェクトを進行していくと以下のような問題が発生した.

\begin{itemize}
\item 実際にプログラムを実装する期間が少ない
\item 完成したプログラムを試行する期間が少ない
\item 巨大な合成数の分解を行いにくい
\end{itemize}

そのため,本プロジェクトでは5月中旬まで全員で最低限の基礎学習を行い,そこから理論班とプログラミング班の2つに分けて作業を行うこととした.また,後期には広報班を作成し,3つの作業を並行で行うこととした.以下にそれぞれの班の課題について述べる.

\begin{description}
 \item[理論班]\mbox{}\\ 
	    ECMについて理解を深め,高速化の新たなアルゴリズムを提案する.
 \item[プログラミング班]\mbox{}\\
	    基礎学習や理論班がまとめたアルゴリズムを実装し高速化を行う.
 \item[広報班]\mbox{}\\
	    ECMについて理解してもらえるようなWebページの作成をする.
\end{description}
\bunseki{伊藤有輝}

\end{document}

%前期活動内容
% 両面印刷する場合は `openany' を削除する
\documentclass[openany,11pt,papersize]{jsbook}

%パッケージの読み込みなど
% 報告書提出用スタイルファイル
\usepackage[final]{funpro}%最終報告書
%\usepackage[middle]{funpro}%中間報告書

% 画像ファイル (EPS, EPDF, PNG) を読み込むために
\usepackage[dvipdfmx]{graphicx,color}

%数式の表示に利用するため
\usepackage{amsmath,amssymb}

%アルゴリズムの表示に利用するパッケージ
\usepackage{algorithm}
\usepackage{algorithmic}

%枠をつけるためのパッケージ
\usepackage{ascmac}

%図の位置調整パッケージ
\usepackage{here}

%付録を作成するためのパッケージ
\usepackage{appendix}

%ドキュメント管理用パッケージ
\usepackage{docmute}


% ここから -->
\usepackage{calc,ifthen}
\newcounter{hoge}
\newcommand{\fake}[1]{\whiledo{\thehoge<70}{#1\stepcounter{hoge}}%
  \setcounter{hoge}{0}}
% <-- ここまで 削除してもよい


% 年度の指定
\thisYear{2016}

% プロジェクト名
\jProjectName{FUN-ECM プロジェクト}

% [簡易版のプロジェクト名]{正式なプロジェクト名}
% 欧文のプロジェクト名が極端に長い(2行を超える)場合は,短い記述を
% 任意引数として渡す.
%\eProjectName[Making Delicious curry]{How to make delicious curry of Hakodate}
\eProjectName{FUN-ECM Project}


% <プロジェクト番号>-<グループ名>
\ProjectNumber{15-A}

% グループ名
\jGroupName{Aグループ}
\eGroupName{A Group}

% プロジェクトリーダ
\ProjectLeader{1014129}{池野竜將}{Ryusuke Ikeno}

% グループリーダ
\GroupLeader  {1014129}{池野竜將}{Ryusuke Ikeno}

% メンバー数
\SumOfMembers{8}
% グループメンバ
\GroupMember  {1}{1014068}{駒ヶ嶺壮}{Sou Komagamine}
\GroupMember  {2}{1014109}{伊藤有輝}{Yuki Ito}
\GroupMember  {3}{1014129}{池野竜將}{Ryusuke Ikeno}
\GroupMember  {4}{1014137}{千葉大樹}{Daiju Chiba}
\GroupMember  {5}{1014164}{橋本和典}{Kazunori Hashimoto}
\GroupMember  {6}{1014168}{山下哲平}{Teppei Yamashita}
\GroupMember  {7}{1014209}{源啓多}{Keita Minamoto}
\GroupMember  {8}{1013150}{亀谷浩也}{Hiroya Kametani}

% 指導教員
\jadvisor{白勢政明,由良文孝}
% 複数人数いる場合はカンマ(,)で区切る.カンマの前後に空白は入れない.
\eadvisor{Masaaki Shirase, Fumitaka Yura}

% 論文提出日
\jdate{2016年7月27日}
\edate{July~27, 2016}


\begin{document}

\chapter{前期活動内容}

プロジェクトが始まった当初,ほぼ全員楕円曲線についての前提知識がなかったため,昨年も前提知識を身に着けるために使われた全員楕円曲線についての資料を全員で輪読し,理解した.その際,理解できなかったところを由良先生,白勢先生に解説してもらった.それにより,楕円曲線法のアルゴリズムを理解するためにあたっての基礎知識を学んだ.その後,プロジェクト全体をプログラムの高速化につながる理論を探し,学習してアルゴリズムをノートにまとめる理論班,理論班がノートにまとめたアルゴリズムをプログラムに実装するプログラミング班に分けてプロジェクトを進めた.

\bunseki{伊藤有輝}

\section{基礎学習}

去年のプログラムを理解するために5月の中旬まではメンバ全員が教授の指導のもとで楕円曲線法のアルゴリズムや基礎知識ついての基礎学習を行った.具体的な内容は以下の通りである.
\begin{description}
 \item[有限体]\mbox{}\\ 
	      $素数pに対し,0からp-1までの整数の集合\mathbb{F}_p=\{0,1,…,p-1\}を有限体と言う.\mathbb{F}_pでは四則演算が可能であり,ECMではこの範囲で考える.$
 \item[Euclidの互除法]\mbox{}\\
		$自然数a,b(a≧b)に対して以下の操作を繰り返し余りが0になるまで行うことによってa,bの最小公倍数を求めるものである.$

\begin{algorithm}[h]                   
\caption{Euclidean Algorithm}
\label{alg E}                          
\begin{algorithmic}                  
\REQUIRE $a,b \in \mathbb{N} , \quad a,b \neq 0,\quad a\ge b$
\ENSURE $\gcd (a,b)$
\WHILE {$b \neq 0$}
\STATE $q \leftarrow a/b$
\STATE $r \leftarrow a\mod b$
\STATE $a \leftarrow b$
\STATE $b \leftarrow r$
\ENDWHILE
\end{algorithmic}
\end{algorithm}
$a,bの最大公約数を\gcd (a,b)と表記できる.$
	
 \item[拡張Euclidの互除法]\mbox{}\\
	$与えられた整数a,b,cに対し,未知数x,yに関する一次方程式ax+by=c$の整数解を求める問題を一次不定方程式という.ここで,$自然数a,bに関する一次不定方程式ax+by=gcd(a,b)を満たす無数の整数x,yは拡張$Euclidの互除法を用いることで効率よく求めることができる.これはEuclidの互除法で行った操作を逆に行うことで解を得る.$gcd(174,69)=3を例にとって考える.$

	\begin{align*} 
		174/69&=2*69+36 \\
		69/36&=1*36+33 \\
		36/33&=1*33+3 \\
		33/3&=11 	
 	\end{align*}
	となるので
	\begin{align*} 
	3&=36-33*1 \\
	&=36-(69-36*1)*1 \\
	&=69*(-1)+36*2 \\
	&=69*(-1)+(174-69*2)*2 \\
	&=174*2+69*(-5)
 	\end{align*}

以上より,$174x+69y=3の解(x,y)=(2,-5)$を得ることができる。有限体$\mathbb{F}_p$において除算$a/b$を計算する場合, $pとb$は互いに素なので, 拡張Euclidの互除法により不定方程式$px + by = 1 の解(x, y)$を求めることができる。このとき$px+ by= 1$となるので, 有限体$\mathbb{F}_p$上では$by = 1となり,両辺をbで割ることで,b^−1 = y$が成立する。したがって$a ÷ b = a×b-1=a × y$と変形することで, 除算を乗算に置き換えて計算できる。プログラムにおいて、除算を乗算に置き換えることは計算量の削減につながるが、今回のプロジェクトではGMPライブラリを用いたことでこれを実装することはなかった。


	
\item[楕円曲線の定義方程式]\mbox{}\\
	$a,b \in \mathbb{F}_pに対してy^2 = x^3 + ax + bで定義される曲線を素体Fp上の楕円曲線という.$

\item[楕円曲線の加算・2倍算]\mbox{}\\
	$(加算) 楕円曲線上のある2点P,Qを通る直線をℓとすると,楕円曲線と直線 \ell の3つ目の交点R’(=P×Q)のx軸に関する対称点をRとする。このとき2点P,Qの和をR=P+Qと定義し、楕円曲線の加算という。$
	
$(2倍算) 楕円曲線上の1点Pで加算を考えるときは2点P,Pの通る直線(=Pの接線)をℓとして考える。この時、楕円曲線と直線 \ell のP以外の交点のx軸に関する対称点をRとしたとき、R=P+P=2Pとできる。これが楕円曲線の2倍算である。
$
	
\item[楕円曲線のスカラー倍]\mbox{}\\
	$点Pと整数mを使用して,mP=P+P+P+P+・・・・+P(m個の和)と表すことができる.これを楕円曲線のスカラー倍という.$

\item[楕円曲線法のアルゴリズム ]\mbox{}\\
	$Nを素因数分解したい合成数とする.\mathbb{Z}/N\mathbb{Z}上で,楕円曲線Eを構成して,点$
	\begin{equation}
	P \in E(\mathbb{Z}/N\mathbb{Z})
	\end{equation}
	$をとる.初めにPの座標を決めてからEを構成しても良い.$
	
	$次に適切なB_1,L=2,3,・・・B_1の最小公倍数とする.LPの計算の過程で生じる点の座標の分母dが\gcd(N,d) \neq 1となるとNの約数を発見できる.$
	
	$最期まで\gcd (N,d)=1ならば,EとPを選びなおしてやり直す.適切なB_1を選ぶことで,ECMは高速な素因数分解法になることが知られている.$
\end{description}
以上のことを基礎学習として学んだ.以下の章ではに2班に分かれた後の理論班の活動内容を記述する.

\bunseki{橋本和典}

\section{理論班}
理論班では新たなアルゴリズムを探し,プログラミング班に新たな高速化手法の提案を行った.以下に具体的な内容を述べる.

\subsection{Twisted Edwards Curveの理解}
ECMの高速化アルゴリズムを実装するにあたって,先人の知恵を得ようと思いインターネットで類似研究の論文を検索し,その論文を解読することによって高速化アルゴリズムをプログラムに導入しようと考えた.その際,Twisted Edwards Curves Revisitedというエドワーズ曲線についての英語の論文が見つかったため,私たちはこの論文を読解することにした.

この論文は,最初に一般的な楕円曲線アルゴリズムより,エドワーズ曲線の方が計算コストは低く,速いスピードで素因数を求めることができるということが説明されており,そのエドワーズ曲線の数学的な理論とプログラム実装のためのアルゴリズムが書かれていた.

エドワーズ曲線については基礎学習で学んでいなかったため,私たちはエドワーズ座標を学習した.その中では射影座標が使用されていた.射影座標とは一般的なの座標(x,y)に対して$x=\frac{X}{Z},y=\frac{Y}{Z}を満たすX,Y,Zを用いて(X,Y,Z)と表す座標であり,射影座標を用いると$ECMアルゴリズムを高速化することができる.具体的な定義は以下の通りである.

\begin{itembox}[l]{射影座標}
\begin{center}
$(X,Y,Z)=(\lambda X, \lambda Y, \lambda Z)=$$(\displaystyle \frac{X}{Z}$,$\displaystyle \frac{Y}{Z}$,1) $(Z\neq0)$
\end{center}
\end{itembox}

理論班では,この拡張エドワーズ座標の理論を学ぼうとしたが,知識が乏しく,わからない変数が出てきたため,アルゴリズムだけを理解し,定義,証明などの理論を理解することはあきらめた.

\bunseki{伊藤有輝}

\subsection{Atkin-Morain ECPP}
\label{sec:ECPP}
次にAtkin-Morain ECPPというECMの初期座標を決定するアルゴリズムの理解に励んだ.昨年度まではECMの初期座標として(2,2)を用いて,素因数分解が完了できなければ(2,3),(2,4)…といったようにY座標を1ずつ動かすようにしていたが,今年度では少しでもを因数を見つける確率を上げることが見込めるAtkin-Morain ECPPを理解することにした.Atkin-Morain ECPPでは新たな楕円曲線$T^2=S^3-8S-32の点を用意し,(S,T)=(12,40)に対してn(S,T)の座標(s,t)$を用いて以下を定める.

\begin{center}
\begin{equation}
\alpha =\cfrac{(s-9)+1}{t+25}  ,  \beta = \cfrac{2\alpha (4\alpha +1)}{8\alpha^2-1}
\end{equation}
\end{center}
これらを用いることによって、素因数分解に用いる楕円曲線の初期座標を求めることができる。具体的には以下の通りである。

\begin{algorithm}[h]                   
\caption{Atkin-Morain ECPP Algorithm}
\label{alg ECPP}                          
\begin{algorithmic}                  
\REQUIRE $\alpha,\beta,s,t,\in \mathbb{N}$
\ENSURE $(X,Y)$
\STATE $(s,t) \leftarrow(12,40)$
\WHILE {Prime factor is not found}
\STATE $\alpha \leftarrow \cfrac{(s-9)+1}{t+25}$
\STATE $\beta \leftarrow \cfrac{2\alpha (4\alpha +1)}{8\alpha^2-1}$
\STATE $d \leftarrow \cfrac{2(2\beta -1)^2-1}{(2\beta -1)^4}$
\STATE $E:x^2+y^2=1+dx^2y^2$
\STATE $X \leftarrow \cfrac{(2\beta -1)(4\beta -3)}{6\beta -4}$
\STATE $Y \leftarrow \cfrac{(2\beta-1)(t^2+50t-2s^3+27s^2-104)}{(t+3s-2)(t+s+16)}$
\STATE Run ECM with $E:x^2+y^2=1+dx^2y^2$ and $(X,Y)$
\STATE $(s,t) \leftarrow 2(s,t)$
\ENDWHILE
\end{algorithmic}
\end{algorithm}

このアルゴリズムを用いると具体的には従来の1.5倍ほど高速化できる見込みであるが、これは論文上のデータである。したがって、後期はプログラム班が実装し、どのくらい速くなるかどうかを検証したいと考えている。

\bunseki{伊藤有輝}

\section{プログラミング班}
プログラミング班では,昨年度のFUN-ECMプロジェクトで作成したECMプログラムをさらに高速化するために,4月から5月にかけて行った全体での基礎学習や,理論班がまとめた理論・アルゴリズムを元にプログラムを変更した.主に,射影座標やextended twisted Edwards coordinatesを用いて乗算・除算を減らすことによって高速化を図った.また,前年度のプログラムの不具合等も改善した.具体的には以下の通りである.

\bunseki{源啓多}

\subsection{座標変換の際の冗長なコストの削減}\label{sec:alg1}
前年度のプロジェクトで作成されたECMプログラムでは,スカラー倍をする際の座標をアフィン座標から射影座標に変換することで計算効率を上昇させていた.このアフィン座標から射影座標への変換は複数回呼び出される為,ECMプログラムの計算コストに影響する.Algorithm \ref{alg:algP}にアルゴリズムを記す.

\begin{algorithm}[H]
\caption{Affine Coordinates to Projective Coordinates (Past ver.)}
\label{alg:algP}                          
\begin{algorithmic}                  
\REQUIRE $(AX,AY)$ is Affine, $(PX,PY,PZ)$ is Projective, $N \ge 2 $
\ENSURE $(PX,PY,PZ)$
\STATE $Z \leftarrow Random(0 \le Z < N)$
\IF {$Z=0$}
\STATE $Z \leftarrow 1$
\ENDIF
\STATE $AX \leftarrow AX \times Z$
\STATE $AY \leftarrow AY \times Z$
\STATE $AX \leftarrow AX \mod N$
\STATE $AY \leftarrow AY \mod N$
\STATE $(PX,PY,PZ) \Leftarrow (AX,AY,Z)$
\end{algorithmic}
\end{algorithm}


前述の冗長部分として乗算が2回と$mod$の計算が2回発生している.プログラミング班では,$Z$の値を1に設定することで乗算と$mod$の計算を省略できると考えた.プログラムを一通り読み直し,問題が発生しないことを確認したのち,新たなアルゴリズムを実装した.Algorithm \ref{alg:algN}に新しいアルゴリズムを示す.

\begin{algorithm}[H]                   
\caption{Affine Coordinates to Projective Coordinates (New ver.)}
\label{alg:algN}                          
\begin{algorithmic}                  
\REQUIRE $(AX,AY)$ is Affine, $(PX,PY,PZ)$ is Projective, $N \le 2 $
\ENSURE $(PX,PY,PZ)$
\STATE $Z \leftarrow 1$
\STATE $(PX,PY,PZ) \Leftarrow (AX,AY,Z)$
\end{algorithmic}
\end{algorithm}

\bunseki{源啓多}

\subsection{Extended twisted Edwards coordinatesの実装}\label{sec:alg2}
前年度のプロジェクトで作成されたECMプログラムでは,twisted Edwards curveを利用している.今回のプログラミング班ではさらにextended twisted Edwards coordinatesを用いた.extended twisted Edwards coordiantesはエドワーズ曲線のスカラー倍を高速化するための座標系であり,以下で定義される補助座標Tを加えた4つの座標でスカラー倍を行う.

\begin{itembox}[l]{Extended twisted Edwards coordinates}
射影座標(X,Y,Z)をに対し,T=$\cfrac{XY}{Z}$という補助座標を加える.これをExtended twisted Edwards coordinatesと呼ぶ.
\begin{center}
$(X,Y,Z) \rightarrow (X,Y,T,Z)$
\end{center}
\end{itembox}
\bunseki{源啓多}

\subsection{楕円曲線の生成法の変更}\label{sec:alg3}
楕円曲線法を利用したECMプログラムは,楕円曲線を生成しその座標を利用し素因数分解を行うプログラムである.また,本プロジェクトで素因数分解しようと試みている合成数は200桁前後のため,1度の試行では素因数分解できないことが多くある.よって,同じ合成数に対して複数回の試行をすることを想定してプログラムを作成する必要がある.前年度のプログラムでは,楕円曲線を生成する際に,Y値をfor文のカウンタを利用して1から順に決めるアルゴリズムを採用していた.そのため,複数回試行した際に同じ曲線を使用してしまうことが多くあり,効率が落ちていたと仮定した.そこで曲線を生成する際に使用しているY値に乱数を使用することとした.

\bunseki{源啓多}

\section{中間発表}

\subsection{準備}
\begin{description}
\item[ポスター]\mbox{}\\
初めに,前年度のプロジェクトで作成されたポスターを参考に構成を決定した.次に,概要,基礎学習,理論班,プログラミング班の4つの項目に分け,作成を分担した.ポスターの作成には「Microsoft PowerPoint」というソフトウェアを使用した.ポスターが完成次第,理論班・プログラミング班でレビューを行い,誤字脱字等を修正した.しかしポスターレビューが不十分だったため,最終的に完成したポスターで誤植が見つかってしまった.

\bunseki{亀谷浩也}

\item[プレゼンテーション資料]\mbox{}\\
本プロジェクトの内容を説明するには,ポスターだけでは足りないと判断しプレゼンテーション資料を作成することに決定した.作成にあたって,まず1名がプレゼンテーションの大まかな流れを作成し,各自作成する章を分担した.プレゼンテーション資料の作成には「Microsoft PowerPoint」というソフトウェアを使用した.また,一度完成したプレゼンテーション資料を先生にレビューしていただき,資料中のグラフの不備や内容についての助言を受けた.それを受け,文章や図の修正を行った.これにより,より見やすいプレゼンテーション資料が完成した.

\bunseki{亀谷浩也}

\item[原稿]\mbox{}\\
前述のプレゼンテーション資料の作成と並行して,発表用の原稿の作成を行った.こちらも1名が大まかな流れを作成し,各自作成する章を分担した.特に楕円曲線法については,何も知らない聴衆でもわかりやすく説明できるように,専門的な用語を最小限にするように注意して作成した.何度か原稿とプレゼンテーション資料を使用しプレゼン練習を行い,伝わりにくい表現や冗長な表現を修正した.
\end{description}

\bunseki{亀谷浩也}

\subsection{発表}
発表は前後半で4人ずつに分かれ,発表を行った.それぞれが自分の担当する部分を読み上げ,その間他の3人は評価アンケート配布や,ポスターに関しての質問に対応した.発表途中にプロジェクターの電源が落ちてしまうというアクシデントがあったが,落ちている間はPCの画面を直接見せることでプレゼンを行い,他の3人で復旧作業を行った.発表後に評価アンケートの集計を行った結果、発表技術は10点中平均7.1点、発表内容は10点中7.5点だった。コメントでは内容を理解していた人と全く理解できない人が分かれていたため、さらに前提知識のない聴衆にも伝わるような内容にしていきたい。

\bunseki{亀谷浩也}

\end{document}

%後期活動内容
% 両面印刷する場合は `openany' を削除する
\documentclass[openany,11pt,papersize]{jsbook}

%パッケージの読み込みなど
% 報告書提出用スタイルファイル
\usepackage[final]{funpro}%最終報告書
%\usepackage[middle]{funpro}%中間報告書

% 画像ファイル (EPS, EPDF, PNG) を読み込むために
\usepackage[dvipdfmx]{graphicx,color}

%数式の表示に利用するため
\usepackage{amsmath,amssymb}

%アルゴリズムの表示に利用するパッケージ
\usepackage{algorithm}
\usepackage{algorithmic}

%枠をつけるためのパッケージ
\usepackage{ascmac}

%図の位置調整パッケージ
\usepackage{here}

%付録を作成するためのパッケージ
\usepackage{appendix}

%ドキュメント管理用パッケージ
\usepackage{docmute}


% ここから -->
\usepackage{calc,ifthen}
\newcounter{hoge}
\newcommand{\fake}[1]{\whiledo{\thehoge<70}{#1\stepcounter{hoge}}%
  \setcounter{hoge}{0}}
% <-- ここまで 削除してもよい


% 年度の指定
\thisYear{2016}

% プロジェクト名
\jProjectName{FUN-ECM プロジェクト}

% [簡易版のプロジェクト名]{正式なプロジェクト名}
% 欧文のプロジェクト名が極端に長い(2行を超える)場合は,短い記述を
% 任意引数として渡す.
%\eProjectName[Making Delicious curry]{How to make delicious curry of Hakodate}
\eProjectName{FUN-ECM Project}


% <プロジェクト番号>-<グループ名>
\ProjectNumber{15-A}

% グループ名
\jGroupName{Aグループ}
\eGroupName{A Group}

% プロジェクトリーダ
\ProjectLeader{1014129}{池野竜將}{Ryusuke Ikeno}

% グループリーダ
\GroupLeader  {1014129}{池野竜將}{Ryusuke Ikeno}

% メンバー数
\SumOfMembers{8}
% グループメンバ
\GroupMember  {1}{1014068}{駒ヶ嶺壮}{Sou Komagamine}
\GroupMember  {2}{1014109}{伊藤有輝}{Yuki Ito}
\GroupMember  {3}{1014129}{池野竜將}{Ryusuke Ikeno}
\GroupMember  {4}{1014137}{千葉大樹}{Daiju Chiba}
\GroupMember  {5}{1014164}{橋本和典}{Kazunori Hashimoto}
\GroupMember  {6}{1014168}{山下哲平}{Teppei Yamashita}
\GroupMember  {7}{1014209}{源啓多}{Keita Minamoto}
\GroupMember  {8}{1013150}{亀谷浩也}{Hiroya Kametani}

% 指導教員
\jadvisor{白勢政明,由良文孝}
% 複数人数いる場合はカンマ(,)で区切る.カンマの前後に空白は入れない.
\eadvisor{Masaaki Shirase, Fumitaka Yura}

% 論文提出日
\jdate{2016年7月27日}
\edate{July~27, 2016}


\begin{document}

\chapter{後期活動内容}
後期の活動は,前期の活動に加えて広報活動を行った.また,理論班には作成したプログラムの検証を行った.
\bunseki{亀谷浩也}

\section{理論班}
\label{sec:theoryresult}
理論班は後期は前期のようにECMプログラムの改善方法の提案は行わず,今年度にプログラミング班が作ったECMプログラムと昨年度のECMプログラムを実行しどれだけ改善したかを検証した.
検証するにあたり,まず今年度はどのような方法で昨年度のプログラムと比較するか考えた。昨年度の報告書を参考にし,昨年度では素数を入力しアルゴリズムが終了するまで時間を計測し比較していたが,今年度はプログラムの処理速度の改善ではなく素因数を発見する確率を上げたため今回はこの方法では検証しなかった。そして,白勢先生のアドバイスに基づいて,検証を行った。各試行に時間がかってしまい検証回数が少なかったので統計としてはデータが少なく信頼性が低いが,数値として結果を表すことができた。

\begin{table}[htb]
  \begin{tabular}{|l|r|r||l|} \hline
    桁数 & 今年度(秒) & 昨年度(秒) & 平均改善率平均 \\ \hline \hline
    20 & 3.400~3.696             & 1.689~2.257             & -82\% \\ \hline
    25 & 10.906~11.323          & 8.528~9.513             & -25\% \\ \hline
    30 & 47.165~51.537          & 44.223~47.599          & -4\% \\ \hline
    35 & 177.932~190.253       & 191.348~200.441       & +6\% \\ \hline
    40 & 686.793~693.397       & 711.594~713.633       & +4\% \\ \hline
    45 & 2682.470~2745.112    & 2653.112~2667.896    & +0\% \\ \hline
    50 & 10173.577~10622.231 & 11763.204~11849.030 & +12\% \\ \hline
  \end{tabular}
\end{table}

20桁から30桁では今年度のプログラムの平均改善率は昨年度より下回る結果になったが35桁以降は徐々に上がっていき最大15%も改善し処理速度が速くなった.よって,巨大な桁数の合成数を素因数分解するのに昨年度より処理時間が速くなった.
\bunseki{亀谷浩也}

\section{プログラミング班}
プログラミング班は前期の活動に引き続き,ECMプログラムの改善を行った.
\bunseki{亀谷浩也}

\subsection{Atkin-Morain ECPPの実装}
後期にまず行ったのはAtkin-Morain ECPPの実装だ.このアルゴリズムは前期中に理論班によって提示されたものだ.詳細なアルゴリズムは\ref{sec:ECPP}に記述した.
\bunseki{亀谷浩也}

\subsection{スカラー倍算の高速化}
Atkin-Morain ECPPを実装後,私たちは楕円曲線上の点のスカラー倍算の高速化に取り組んだ.スカラー倍算はECMのアルゴリズムの中で最も計算量の多い箇所の為,高速化が期待できると考えたからだ.そのために,移動窓法(sliding/moving window method)を実装した。移動窓法はバイナリ法(binary method)やm進展開法(window method),符号付きm進展開窓法などと比較される楕円曲線演算の高速化手法だ。移動窓法の理解に際し,バイナリ法及びm進展開法について学んだので,まずはこれらについて記述する。
\bunseki{源啓多}

\subsubsection{バイナリ法}
バイナリ法は楕円曲線上の点$Pをk倍した点kPを求める際に,k$を2進展開することで,高速なスカラー倍算を実現する手法である。昨年度から引き継いだプログラムでは,このアルゴリズムが採用されていた.バイナリ法の詳細なアルゴリズムを以下に示す。

\begin{algorithm}[H]                   
\caption{binary method}
\label{alg:algB}                          
\begin{algorithmic}                  
\REQUIRE $P, n$bit integer $k = {\displaystyle \Sigma_{i=0 \rightarrow n-1}}k_j 2^j$,$k_j \in \{0,1\}$
\ENSURE $Q = kP$
\STATE $P_1 \leftarrow P$
\FOR {$i = 2$ to $m-1$}
\STATE $P_i \leftarrow P_{i-1} + P$
\ENDFOR
\STATE $Q \leftarrow 0$
\FOR {$j=d-1$ to $0$ by $-1$}
\STATE $Q \leftarrow mQ$
\STATE $Q \leftarrow Q+P_{k_j}$
\ENDFOR
\end{algorithmic}
\end{algorithm}

バイナリ法を用いなかった場合,$kPを計算する手順はP+P+P+…+Pであり,これには加算がk-1$回必要である。対して,バイナリ法を用いた場合は加算と2倍算がそれぞれ回で済む。したがって,バイナリ法を採用することで大きいkに対して高速にスカラー倍算を行うことができる。

\bunseki{源啓多}

\subsubsection{m進展開法}
 $m$進展開法では,バイナリ法を応用して,2進展開ではなく$m進展開を行っている。m進展開の容易さから,mは2^r(r\in Z, r>Z)のような値であることが多い。事前計算として2P, 3P, ・・・, (m-1)Pを計算する必要があるが,
であるとき,rビット単位で計算を行うことができるので,高速化につながる。以下は,m進展開法のアルゴリズムである。$

\begin{algorithm}[H]                   
\caption{window method}
\label{alg:algW}                          
\begin{algorithmic}                  
\REQUIRE $P, k = \Sigma_{i=0\rightarrow d-1}k_j m^j, k_j\{0,1, ... , m-1\}$
\ENSURE $Q = kP$
\STATE $P_1 \leftarrow P$
\FOR {$i =2$ to $m-1$}
\STATE $P_i \leftarrow P_{i-1} + P$
\ENDFOR
\STATE $Q \leftarrow 0$
\FOR {$j=d-1$ to $0$ by $-1$}
\STATE $Q \leftarrow mQ$
\STATE $Q \leftarrow Q+P_{k_j}$
\ENDFOR
\end{algorithmic}
\end{algorithm}

\bunseki{源啓多}

\subsubsection{移動窓法}
 $m進展開法をさらに発展させたのが移動窓法である。移動窓法では,計算を行う単位をr$ビットに固定しておらず,末尾のビットが1かつ$r$ビット以下で最長になるような単位で計算を行う。末尾のビットが1であるようにすることで,事前計算の量が$m$進展開法に比べて半分で済むことが移動窓法の特徴である。移動窓法のアルゴリズムを以下に示す。
 
\begin{algorithm}[H]                   
\caption{moving/sliding window method}
\label{alg:algW}                          
\begin{algorithmic}                  
\REQUIRE $P, k=\Sigma_{i=0 \rightarrow n-1}k_j 2^J,k_j \in \{0,1\}$
\ENSURE $Q=kP$
\STATE $P_1 \leftarrow P$
\STATE $P_2 \leftarrow 2P$
\FOR {$i=1$ to $2^{r-1} -1$}
\STATE $P_{2i+1} \leftarrow P_{2i-1}+P_2$
\ENDFOR
\STATE $j \leftarrow n-1$
\STATE $Q \leftarrow 0$
\WHILE {$j \geq 0$}
\IF {$k_j = 0$}
\STATE $Q \leftarrow 2Q$
\STATE $j \leftarrow j-1$
\ELSE
\STATE $t= \min \{ j-t+1 \leq r AND k_t = 1 \}$
\STATE $h_j \leftarrow ( k_j, k_{j-1},  \cdots , k_t)_2$
\STATE $Q \leftarrow [2^{j-t+1}]Q + P_{h_j}$
\STATE $j \leftarrow t-1$
\ENDIF
\ENDWHILE
\end{algorithmic}
\end{algorithm}

他に採用するアルゴリズムとして,移動窓法の実装後にMontgomery ladderが挙がったが,実装・比較する期間を設けられなかったので来年度への課題とする。

\subsection{Stage2}
前述したようなアルゴリズムを調査した結果,現状のアルゴリズムを改善するより新たなアルゴリズムを導入することが良いと考え,後期ではStage2というアルゴリズムを実装した.まず,Stage2の前提として,Stage1を説明する.Stage1は,Pのk倍,すなわちkPを計算する過程である.具体的には,以下のような手順でkを決定し,計算を行っている.

\begin{itembox}[l]{Stage1}
\begin{center}
\[
k = \prod_{2 \leq p \leq B_1, p \in \mathbb{P}} p^{\lfloor \log{p} B_1 \rfloor}
\]
\end{center}
\end{itembox}

前述の式で求めたkを利用してkPを計算し,kPのx座標と合成数の最大公約数をとる.その結果最大公約数が1でなければ素因数分解が成功したことになる.しかし,Stage1だけでは,B1より大きい素数を1つだけkPにかけていれば,素因数が求まった,ということが起こりうる.Stage2は,これの頻度を減らすためのアルゴリズムである.Stage2のための新たなパラメータB2(>B1)を設定し,B1より大きく,B2以下の素数p’それぞれをStage1の計算結果kPにかけ,それぞれのp’(kP)のx座標と合成数Nの最大公約数を計算する.もし計算結果が1でなければ,素因数が求められたことになる.
\bunseki{源啓多}

\subsubsection*{基本的なStage2の実装}

\begin{algorithm}[H]                   
\caption{Basic ECM Algorithm}
\label{alg:B}                          
\begin{algorithmic}                  
\REQUIRE $N$ is composite number, $E$ is elliptic curve, $P = (x_0, y_0, Z_0) \in E(Z_n)$ is initial point, $B_1$ is smoothness bound for Phase 1, $B_2$ is smoothness bound for Phase 2, $B2 \ge B1.$
\ENSURE $q$ is factor of $N$, $1 \le q \leq N$, or FAIL.
\STATE \bfseries{Phase 1.}
\STATE $k \gets \prod_{p \leq B_1} p^{\log{p} B_1}$
\STATE $Q_0 \gets kP_0$
\STATE $q \gets \gcd(z_{Q_0},N)$
\IF {$q \ge 1$}
\STATE return $q$
\ELSE
\STATE go to Phase 2
\ENDIF
\STATE \bfseries{Phase 2.}
\STATE $d \gets 1$
\FOR {each prime $p = B_1$ to $B_2$} 
\STATE $(x_{pQ_0},y_{pQ_0},z_{pQ_0}) \gets pQ_0$
\STATE $d \gets d*Z_{pQ_0} (mod N)$
\ENDFOR
\STATE $q \gets \gcd(d,N)$
\IF {$q \ge 1$}
\STATE return $q$
\ELSE
\STATE return FAIL
\ENDIF
\end{algorithmic}
\end{algorithm}

\bunseki{源啓多}

\section{広報班}
広報班では,プロジェクト活動や楕円曲線法の解説を,外部に発信することにした。発信をする方法として,FUN-ECMのWEBページを制作することにした。WEBサイトを制作するにあたって後期の活動から広報班を新たに結成し,活動を行った。発信する対象は主に情報系の大学生とし,学部一年生の知識でも理解できるように楕円曲線法とその周辺の理論的な基礎知識を表記した。また,来年度以降の活動や,専門知識をもった人にむけて今年得た知識やプログラムで変更した点など,今年度の専門的な活動内容も表記した。
\bunseki{亀谷浩也}

\subsection{動機}
FUN-ECMは今年で3年目であるが,毎年ECMのプログラムの改良を重ねる活動であるため毎年外部への露出が少なく,継続性の強いプロジェクトであることから私たちは今年度ならではの対外的な新規活動をしたいと考えた。新規的な活動をするにあたり,未来大生でもECMについて知らない方が多いことからより多くの人にECMを伝えられるように広報的活動を行うことにした。また,中間発表での意見でプレゼンでは理解しにくかったとの意見を頂いたことからよりわかりやすく伝えられ,手軽にみることができる媒体としてwebページを採用した。

\bunseki{駒ヶ嶺王}

\subsection{Webページの構成と内容}
\subsubsection{Top}
本サイトのトップページ。SNSの共有ボタン,活動写真のスライダーやFUN-ECMとは何かに関する簡単な説明がある。また,ここでは来年度の活動のためのアンケートページも設置した。一番下にあるボタンから各ページに飛ぶことができる。

\bunseki{亀谷浩也}

\subsubsection{About}
広報班が活動目的としていた内容のコンテンツである。ECMの基礎理論の説明ページ,FUN-ECMの活動目的,今年度のECMプログラムの解説ページの3つに分かれている。章ごとに分けて詳細的な説明を行い,段階的に読み進めることができるようにした。また,重要な箇所での色の変更やgif画像を挿入するなどしてより理解しやすいように工夫した。ECMプログラムの解説ページでは今年度のプログラムのソースコードの一部を掲載し,すべてのソースコードについてはURLからgithubのページに飛ぶことができるようにした。

\bunseki{亀谷浩也}

\subsubsection{History}
2016年度の活動月表を掲載した。前期活動と後期活動の2つに分けて,後期活動の中には2016年度の全体成果についても記載した。どの班がどのような活動をしたかについて簡単にわかるようにした。

\bunseki{亀谷浩也}

\subsubsection{Link}
リンク集。本プロジェクトで使用したECM-NETや本学のホームページ等を掲載した。

\bunseki{亀谷浩也}

\subsection{Webページ内のファイルの説明}

\subsubsection{bootstrap.css}
Webサイトやwebアプリケーションを作成するためのwebアプリケーションフレームワークである。Class属性を指定するだけで簡単に豊富なスタイルを指定することができるファイル。

\subsubsection{bootstrap.min.css}
bootstrap.cssの圧縮版であるファイル。読み込みを早くしたい時などはbootstrap.cssではなくこちらを使用する。

\subsubsection{jquery.bxslider.css}
スライドショー形式で表示された画像のスタイルを指定しているファイル。

\subsubsection{jsxgraph.css}
javascriptを用いた楕円曲線のグラフのスタイルをしているファイル。

\subsubsection{original.css}
上記以外のスタイルを指定している。レイアウトのために私たちが設定した

\subsubsection{bootstrap.js}
bootstrapのjavascriptの部分を動かすためのファイル。後述するjQueryのファイルを先に読み込まなければ機能しない。

\subsubsection{bootstrap.min.js}
 bootstrap.jsの圧縮版であるファイル。読み込みを早くしたい時などはbootstrap.jsではなくこちらを使用する。

\subsubsection{jquery.bxslider.min.js}
画像をスライドショー形式で表示するための関数を組み込んでいるファイルである。

\subsubsection{jquery.js}
jQueryをWEBページ内に組み込むためのファイル。jQueryとはjava scriptをより扱いやすくしたファイルであり,本来であれば複雑なプログラムの記述もjQueryを用いることで簡易的に記述することができる。

\subsubsection{jquery.min.js}
jquery.jsの圧縮版であるファイル。読み込みを早くしたい時などはjquery.jsではなくこちらを使用する。

\subsubsection{jsxgraphcore.js,GeonextReader.js}
webページ内のグラフを動かすためのjavascriptが記述されたファイル。

\subsubsection{ecm.html}
FUN-ECMウェブサイトの”ECMとは”	について書かれているhtmlファイルである。「ECMとは何か」について,基礎知識としてmodNの説明や点同士に置けるか山野に倍残についての説明が記述されている。

\subsubsection{ecm1.html}
FUN-ECMウェブサイトの”活動目的”について書かれているhtmlファイルである。FUN-ECMがなぜ素因数分解をするのか,またECM-NETとは何かについて記述されている。

\subsubsection{index.html}
FUN-ECMウェブサイトの”トップページ”について書かれているhtmlファイルである。FUN-ECMの活動風景の写真や名前の由来,各ページへのリンクが記述されている。

\subsubsection{link.html}
FUN-ECMウェブサイトの”リンク”について書かれているhtmlファイルである。ECM-NETやSTUDIO KAMADAなど,ECMに関係するサイトやECMについての説明がされているサイト,またFUN-ECMの教授へのリンクも記述されている。

\subsubsection{log.html}
FUN-ECMウェブサイトの”前期活動”について書かれているhtmlファイルである。4月から8月までのFUN-ECMの活動記録が記述されている。

\subsubsection{log2.html}
FUN-ECMウェブサイトの”後期活動”について書かれているhtmlファイルである。9月から1月までのFUN-ECMの活動記録が記述されている。

\subsubsection{member.html}
FUN-ECMウェブサイトの”メンバー紹介”について書かれているhtmlファイルである。メンバー全員の名前と班,一言が記述されている

\subsubsection{program.html}
 FUN-ECMウェブサイトの”プログラムについて”について書かれているhtmlファイルである。FUNECMプログラムの使い方やECMの基本的な実装,さらにその他のECMに関する専門的な知識やそのプログラムが記述されている。
 
\bunseki{亀谷浩也}

\subsection{展望}
私たちは当初,情報大学生の学部1年生でも私たちの活動を理解することのできるような解説ページを設けるという一つの目標を立て,実際にECMについての大まかな流れを難しいと思われる部分を砕きつつ解説したページを作成した。そして最終発表会において私たちの発表を見てもらった方にウェブページにあるアンケートで感想を答えてもらった。しかし,アンケートに答えてもらった方の母数が少なかったため,来年度はアンケートに答えてもらう人数を今年度より増やすことによって正確な理解度の調査を行い,その結果を使いウェブページの改善を行いたい。

\section{成果発表}
成果発表会では,前期に行った中間発表のレビューを元に改善をした.レビューでは,内容が理解できた人とできていない人が分かれていたため,さらに前提知識のない聴衆にも伝わるような内容を目指した.

\subsection{準備}
\begin{description}
\item[ポスター]\mbox{}\\
後期の活動では,3つの活動を並行して行っていたため,ポスターを前期に比べて1枚増やし3枚で構成を考えた.次にメインポスターとサブポスターを分け,メインは全体の活動を大まかに伝える,サブポスターはそれぞれの活動を具体的に伝えるという目標を設定し,各自作成をした.ポスターの作成には前年度に引き続き「Microsoft PowerPoint」というソフトウェアを利用した.ポスターが完成次第,グループごとにレビューを行い,誤字脱字を修正した.
\bunseki{亀谷浩也}

\item[プレゼンテーション資料]\mbox{}\\
前期の中間発表のレビューでは,伝わった人と伝わらなかった人が分かれていたため,前提知識が殆どない聴衆でもわかりやすくなるように,専門的な用語を最小限にするように注意し,最も大事でなところを枠で囲み,その中身を見るだけで大まかな内容を理解できるようにした.また,楕円曲線法についての説明では,例示を多く含めることで数学に抵抗のある人でも触れやすいようにした.加えて,長い数式に関しては説明を省きスライドに表示するだけにとどめ,数学が苦手だというかたの抵抗を減らすようにした.作成には前年度に引き続き「Microsoft PowerPoint」というソフトウェアを利用した.
\bunseki{亀谷浩也}

\end{description}

\subsection{発表}
発表は,前後半で4人ずつに分かれて行った.前半は3人がそれぞれの担当について発表を行い,1人がポスターの前で質問を受けたり,評価シートを配るという配役で行った.後半は4人がそれぞれの担当について発表を行い,発表を行っていない1人が他の作業を行った.前期の発表中にプロジェクターの電源が落ちてしまうアクシデントがあった為,そのようなアクシデントに対応するために,PCでプレゼンテーションを行いながら復旧作業を行うことを決めた.発表後に評価アンケートの集計を行った結果,発表技術は10点中平均7.1点,発表内容は10点中7.9点だった.共に前期の評価よりも点数が上昇しており,発表の工夫の効果が表れていることが確認できた.しかし,いくつかのコメントに内容を省きすぎている,数式の説明をしないのはよくない,などの意見もあった.

\bunseki{亀谷浩也}
\end{document}


%インターワーキング
% 両面印刷する場合は `openany' を削除する
\documentclass[openany,11pt,papersize]{jsbook}

%パッケージの読み込みなど
% 報告書提出用スタイルファイル
\usepackage[final]{funpro}%最終報告書
%\usepackage[middle]{funpro}%中間報告書

% 画像ファイル (EPS, EPDF, PNG) を読み込むために
\usepackage[dvipdfmx]{graphicx,color}

%数式の表示に利用するため
\usepackage{amsmath,amssymb}

%アルゴリズムの表示に利用するパッケージ
\usepackage{algorithm}
\usepackage{algorithmic}

%枠をつけるためのパッケージ
\usepackage{ascmac}

%図の位置調整パッケージ
\usepackage{here}

%付録を作成するためのパッケージ
\usepackage{appendix}

%ドキュメント管理用パッケージ
\usepackage{docmute}


% ここから -->
\usepackage{calc,ifthen}
\newcounter{hoge}
\newcommand{\fake}[1]{\whiledo{\thehoge<70}{#1\stepcounter{hoge}}%
  \setcounter{hoge}{0}}
% <-- ここまで 削除してもよい


% 年度の指定
\thisYear{2016}

% プロジェクト名
\jProjectName{FUN-ECM プロジェクト}

% [簡易版のプロジェクト名]{正式なプロジェクト名}
% 欧文のプロジェクト名が極端に長い(2行を超える)場合は,短い記述を
% 任意引数として渡す.
%\eProjectName[Making Delicious curry]{How to make delicious curry of Hakodate}
\eProjectName{FUN-ECM Project}


% <プロジェクト番号>-<グループ名>
\ProjectNumber{15-A}

% グループ名
\jGroupName{Aグループ}
\eGroupName{A Group}

% プロジェクトリーダ
\ProjectLeader{1014129}{池野竜將}{Ryusuke Ikeno}

% グループリーダ
\GroupLeader  {1014129}{池野竜將}{Ryusuke Ikeno}

% メンバー数
\SumOfMembers{8}
% グループメンバ
\GroupMember  {1}{1014068}{駒ヶ嶺壮}{Sou Komagamine}
\GroupMember  {2}{1014109}{伊藤有輝}{Yuki Ito}
\GroupMember  {3}{1014129}{池野竜將}{Ryusuke Ikeno}
\GroupMember  {4}{1014137}{千葉大樹}{Daiju Chiba}
\GroupMember  {5}{1014164}{橋本和典}{Kazunori Hashimoto}
\GroupMember  {6}{1014168}{山下哲平}{Teppei Yamashita}
\GroupMember  {7}{1014209}{源啓多}{Keita Minamoto}
\GroupMember  {8}{1013150}{亀谷浩也}{Hiroya Kametani}

% 指導教員
\jadvisor{白勢政明,由良文孝}
% 複数人数いる場合はカンマ(,)で区切る.カンマの前後に空白は入れない.
\eadvisor{Masaaki Shirase, Fumitaka Yura}

% 論文提出日
\jdate{2016年7月27日}
\edate{July~27, 2016}


\begin{document}

\chapter{プロジェクト内のインターワーキング}
\begin{itemize}
\item 池野竜將(プロジェクトリーダー・プログラミング班)
 \begin{enumerate}
 \renewcommand{\labelenumi}{(\arabic{enumi})}
 \item 楕円曲線法の基礎を学んだ.
 \item 大まかな作業スケジュールを作成し,進捗管理を行った.
 \item 源と協力して前年度のECMプログラムを理解した.
 \item 源のコーディング作業にアドバイスをした.
 \item 理論班からのプログラミング班に関しての質問に回答し,必要があれば聞かれた内容を源に伝えた.
 \item 中間発表会に向けて,プレゼンテーション資料・原稿の原案を作成した.
 \item 中間発表会に向けて,「プログラミング班」の部分のプレゼンテーション資料を作成した.
 \end{enumerate}
 
\item 源啓多(プログラミング班)
 \begin{enumerate}
 \renewcommand{\labelenumi}{(\arabic{enumi})}
 \item 楕円曲線法の基礎を学んだ.
 \item 池野と協力して前年度のECMプログラムを理解した.
 \item ECMプログラムのバージョン管理の為,Gitを学んだ.
 \item 前年度のECMプログラムの実装上のミス(\ref{sec:alg1})を改善した.
 \item ECMプログラム改善のために,新たなアルゴリズム(\ref{sec:alg2}, \ref{sec:alg3})の実装を行った.
 \item 中間発表会に向けて,プログラミング班のプレゼンテーション資料・原稿を作成した.
 \item Stage2の解読・実装をいち早く進めた.
 \item 解析班の作業を助けるためのマクロを作成した.
 \item 広報班に協力し,ウェブページの作成の手助けをした.
 \end{enumerate}
 
\item 山下哲平(理論班)
 \begin{enumerate}
 \renewcommand{\labelenumi}{(\arabic{enumi})}
 \item 楕円曲線法の基礎を学んだ.
 \item 伊藤・駒ヶ嶺と協力してEdwards Curveを利用したECMアルゴリズムの読解を行い,プログラミング班に提案を行った.
 \item 伊藤・駒ヶ嶺と協力してAtkin-Morain ECPPアルゴリズムの理解に取り組んだ.
 \item 中間発表会に向けて,「背景」の部分についてポスターをを作成した.
 \item プログラミング班の要請でプログラムの速度について簡易的な検証を行った.
 \item 伊藤と協力してウェブページの基本的な要素を作成した.
 \item 最終報告書の広報班ページを作成した.
 \end{enumerate}
 
\item 伊藤有輝(理論班)
 \begin{enumerate}
 \renewcommand{\labelenumi}{(\arabic{enumi})}
 \item 楕円曲線法の基礎を学んだ.
 \item 駒ヶ嶺と協力して,エドワーズ曲線の式が導き出される過程を学んだ.
 \item 駒ヶ嶺・山下と協力し,Edwards Curveを利用したECMアルゴリズムの読解を行った.
 \item 駒ヶ嶺・山下と協力し,Atkin-Morain ECPPアルゴリズムの理解に取り組み,プログラミング班に提案を行った.
 \item 中間発表会に向けて,「理論班」の部分のプレゼンテーション資料を作成した.
 \item 源・池野と協力し,Githubの使い方を理解して広報班に伝えた.
 \item 山下と協力してウェブページの基本的な要素を作成した.
 \end{enumerate}
 
\item 駒ヶ嶺壮(理論班)
 \begin{enumerate}
 \renewcommand{\labelenumi}{(\arabic{enumi})}
 \item 楕円曲線法の基礎を学んだ.
 \item 伊藤と協力して,エドワーズ曲線の式が導き出される過程を学んだ.
 \item 山下・伊藤と協力してEdwards Curveを利用したECMアルゴリズムの読解を行った.
 \item 山下・伊藤と協力してAtkin-Morain ECPPアルゴリズムの理解に取り組んだ.
 \item 中間発表会に向けて,「理論班」の部分のポスターを作成した.
 \item 広報班のウェブページ作成のため,過去の作業ログを見直し,まとめた.
 \end{enumerate}
 
\item 橋本和典(理論班)
 \begin{enumerate}
 \renewcommand{\labelenumi}{(\arabic{enumi})}
 \item 楕円曲線法の基礎を学んだ.
 \item 千葉・亀谷と協力して入門書を読み,基礎学習を行った.
 \item 亀谷と協力して基礎学習を簡潔にまとめた解説ノートを作成した.
 \item 中間発表会に向けて,千葉・亀谷と協力して来るであろう質問を予測して対策を行った.
 \item ECMの改善に直結するような文献を探した。
 \item 中間発表会に向けて、ポスターの「理論班」の章を英訳した。
 \item 理論班で検証を行う際に,管理者として中心となって作業した.
 \item 行った検証の結果をまとめ,グラフ化して見やすくした.
 \end{enumerate}
 
\item 千葉大樹(理論班)
 \begin{enumerate}
 \renewcommand{\labelenumi}{(\arabic{enumi})}
 \item 楕円曲線法の基礎を学んだ.
 \item 亀谷・橋本と協力して入門書を読み,基礎学習を行った.
 \item 中間発表会に向けて,ECMについての英論文から重要な単語を抜粋し解説した.
 \item 中間発表会に向けて,亀谷・橋本と協力して来るであろう質問を予測して対策を行った.
 \item 中間発表会に向けて、ポスターの「プログラミング班」の章を英訳した。
 \item 理論班で検証を行う際に,実際にプログラムを動かし,データを全体に共有した.
 \end{enumerate}
 
\item 亀谷浩也(理論班)
 \begin{enumerate}
 \renewcommand{\labelenumi}{(\arabic{enumi})}
 \item 楕円曲線法の基礎を学んだ.
 \item 橋本・千葉と協力して入門書を読み,基礎学習を行った.
 \item 橋本と協力して,基礎学習を簡潔にまとめた解説ノートを作成した.
 \item 中間発表会に向けて,橋本・千葉と協力して来るであろう質問を予測して対策を行った.
 \item 中間発表会に向けて、ポスターの「概要・基礎学習」の章を英訳した。
 \item 理論班で検証を行う際に,データの管理やまとめを手伝い,橋本の補佐として活動した.
 \end{enumerate}
 
\end{itemize}

\bunseki{池野竜將}

\end{document}


%前期活動成果
% 両面印刷する場合は `openany' を削除する
\documentclass[openany,11pt,papersize]{jsbook}

%パッケージの読み込みなど
% 報告書提出用スタイルファイル
\usepackage[final]{funpro}%最終報告書
%\usepackage[middle]{funpro}%中間報告書

% 画像ファイル (EPS, EPDF, PNG) を読み込むために
\usepackage[dvipdfmx]{graphicx,color}

%数式の表示に利用するため
\usepackage{amsmath,amssymb}

%アルゴリズムの表示に利用するパッケージ
\usepackage{algorithm}
\usepackage{algorithmic}

%枠をつけるためのパッケージ
\usepackage{ascmac}

%図の位置調整パッケージ
\usepackage{here}

%付録を作成するためのパッケージ
\usepackage{appendix}

%ドキュメント管理用パッケージ
\usepackage{docmute}


% ここから -->
\usepackage{calc,ifthen}
\newcounter{hoge}
\newcommand{\fake}[1]{\whiledo{\thehoge<70}{#1\stepcounter{hoge}}%
  \setcounter{hoge}{0}}
% <-- ここまで 削除してもよい


% 年度の指定
\thisYear{2016}

% プロジェクト名
\jProjectName{FUN-ECM プロジェクト}

% [簡易版のプロジェクト名]{正式なプロジェクト名}
% 欧文のプロジェクト名が極端に長い(2行を超える)場合は,短い記述を
% 任意引数として渡す.
%\eProjectName[Making Delicious curry]{How to make delicious curry of Hakodate}
\eProjectName{FUN-ECM Project}


% <プロジェクト番号>-<グループ名>
\ProjectNumber{15-A}

% グループ名
\jGroupName{Aグループ}
\eGroupName{A Group}

% プロジェクトリーダ
\ProjectLeader{1014129}{池野竜將}{Ryusuke Ikeno}

% グループリーダ
\GroupLeader  {1014129}{池野竜將}{Ryusuke Ikeno}

% メンバー数
\SumOfMembers{8}
% グループメンバ
\GroupMember  {1}{1014068}{駒ヶ嶺壮}{Sou Komagamine}
\GroupMember  {2}{1014109}{伊藤有輝}{Yuki Ito}
\GroupMember  {3}{1014129}{池野竜將}{Ryusuke Ikeno}
\GroupMember  {4}{1014137}{千葉大樹}{Daiju Chiba}
\GroupMember  {5}{1014164}{橋本和典}{Kazunori Hashimoto}
\GroupMember  {6}{1014168}{山下哲平}{Teppei Yamashita}
\GroupMember  {7}{1014209}{源啓多}{Keita Minamoto}
\GroupMember  {8}{1013150}{亀谷浩也}{Hiroya Kametani}

% 指導教員
\jadvisor{白勢政明,由良文孝}
% 複数人数いる場合はカンマ(,)で区切る.カンマの前後に空白は入れない.
\eadvisor{Masaaki Shirase, Fumitaka Yura}

% 論文提出日
\jdate{2016年7月27日}
\edate{July~27, 2016}


\begin{document}

\chapter{前期活動成果}
本プロジェクトでは,理論班で理解することに成功した高速化アルゴリズムをプログラミング班に伝え,プログラミング班がそのアルゴリズムを実装することによりECMプログラムを作成した.

\bunseki{千葉大樹}

\section{理論班}

理論班は,活動内容で示した,エドワーズ曲線においての射影座標を用いたスカラー倍楕円曲線プログラムでの変数の点の与え方のアルゴリズムの改善点を発見した.乗算の回数,除算の回数が減少したことにより素因数を発見する効率が理論上1.5倍減少したが,実装前との計算コストの実数値の比較についてはまだできていない.また,Atkin-Morain ECPPのアルゴリズムの理解をすることに成功した.これを実装することにより,ECMによって素因数pが見つかる確率は,位数があらかじめ小さな因数dを持つ曲線のみを使用した場合,ランダムに動く部分のサイズがpからp=dに減少するため因数分解に成功する確率を高めることができる.しかし,Atkin-Morain ECPPの理論については理解することができなかった.そのため,プログラミング班にはAtkin-Morain ECPPの実装のためのアルゴリズムを書き起こしレポート用紙を渡すことにより,ECM USING EDWARDS CURVEの読解を終了した.

\bunseki{駒ヶ嶺壮}

\section{プログラミング班}
プログラミング班では,新たなアルゴリズムを実装し,理論上は\ref{tab:cost}のように計算量が減少することが分かった.詳細な実験は行っておらず有意な差があるかどうかは確認できていない.だが,実際に素因数分解を行った結果,処理が早くなっていることが確認できた.

\begin{table}
\begin{center}
\caption{昨年度と今年度のプログラムの計算コストの比較}\label{tab:cost}
\begin{tabular}{ccc}
\hline
& 2倍算 & 2倍算→加算\\
\hline
昨年度 & 3{\bf M}+4{\bf S}+1{\bf D}\footnotemark & 13{\bf M}+5{\bf S}+3{\bf D}\\
今年度 & 3{\bf M}+4{\bf S}+1{\bf D} & 12{\bf M}+4{\bf S}+1{\bf D}\\
\hline
\end{tabular}
\end{center}
\end{table}
\footnotetext{{\bf M}:乗算,{\bf S}:2乗算,{\bf D}:楕円曲線の係数a,dを用いた乗算}

また,実際に巨大な合成数を分解し,昨年度のプログラムとの性能を比較することにした.評価するにあたって,2015年度に作成されたプログラムでテストに使用されていた合成数$10^{306}+1$を素因数分解することで,以前のプログラムとの比較をすることとした.2015年度のプログラムでこの合成数を分解した結果,発見されたもっとも大きな素因数は157538980319816607(21桁)であった.同様に今年改善されたプログラムで分解した結果,発見されたもっとも大きな素因数は112544281755782732673671367061(30桁)であり,より大きな素因数を見つけることができるように改善された.

\bunseki{源啓多}

\end{document}

%後期活動成果
% 両面印刷する場合は `openany' を削除する
\documentclass[openany,11pt,papersize]{jsbook}

%パッケージの読み込みなど
% 報告書提出用スタイルファイル
\usepackage[final]{funpro}%最終報告書
%\usepackage[middle]{funpro}%中間報告書

% 画像ファイル (EPS, EPDF, PNG) を読み込むために
\usepackage[dvipdfmx]{graphicx,color}

%数式の表示に利用するため
\usepackage{amsmath,amssymb}

%アルゴリズムの表示に利用するパッケージ
\usepackage{algorithm}
\usepackage{algorithmic}

%枠をつけるためのパッケージ
\usepackage{ascmac}

%図の位置調整パッケージ
\usepackage{here}

%付録を作成するためのパッケージ
\usepackage{appendix}

%ドキュメント管理用パッケージ
\usepackage{docmute}


% ここから -->
\usepackage{calc,ifthen}
\newcounter{hoge}
\newcommand{\fake}[1]{\whiledo{\thehoge<70}{#1\stepcounter{hoge}}%
  \setcounter{hoge}{0}}
% <-- ここまで 削除してもよい


% 年度の指定
\thisYear{2016}

% プロジェクト名
\jProjectName{FUN-ECM プロジェクト}

% [簡易版のプロジェクト名]{正式なプロジェクト名}
% 欧文のプロジェクト名が極端に長い(2行を超える)場合は,短い記述を
% 任意引数として渡す.
%\eProjectName[Making Delicious curry]{How to make delicious curry of Hakodate}
\eProjectName{FUN-ECM Project}


% <プロジェクト番号>-<グループ名>
\ProjectNumber{15-A}

% グループ名
\jGroupName{Aグループ}
\eGroupName{A Group}

% プロジェクトリーダ
\ProjectLeader{1014129}{池野竜將}{Ryusuke Ikeno}

% グループリーダ
\GroupLeader  {1014129}{池野竜將}{Ryusuke Ikeno}

% メンバー数
\SumOfMembers{8}
% グループメンバ
\GroupMember  {1}{1014068}{駒ヶ嶺壮}{Sou Komagamine}
\GroupMember  {2}{1014109}{伊藤有輝}{Yuki Ito}
\GroupMember  {3}{1014129}{池野竜將}{Ryusuke Ikeno}
\GroupMember  {4}{1014137}{千葉大樹}{Daiju Chiba}
\GroupMember  {5}{1014164}{橋本和典}{Kazunori Hashimoto}
\GroupMember  {6}{1014168}{山下哲平}{Teppei Yamashita}
\GroupMember  {7}{1014209}{源啓多}{Keita Minamoto}
\GroupMember  {8}{1013150}{亀谷浩也}{Hiroya Kametani}

% 指導教員
\jadvisor{白勢政明,由良文孝}
% 複数人数いる場合はカンマ(,)で区切る.カンマの前後に空白は入れない.
\eadvisor{Masaaki Shirase, Fumitaka Yura}

% 論文提出日
\jdate{2016年7月27日}
\edate{July~27, 2016}


\begin{document}

\chapter{後期活動成果}

\section{理論班}
検証するにあたり、まず今年度はどのような方法で昨年度のプログラムと比較するか考えた。昨年度の報告書を参考にし、昨年度では素数を入力しアルゴリズムが終了するまで時間を計測し比較していたが、今年度はプログラムの処理速度の改善ではなく素因数を発見する確率を上げたため今回はこの方法では検証しなかった。おそして、白勢先生にアドバイスをいただき、そのアドバイスに基づいて、検証を行った。統計に関しては検証に時間がかかり検証回数が少なかったので統計としてはデータが少なく信頼性が低いが、数値として結果を表すことができた。
\bunseki{橋本和典}

\section{プログラム班}
後期の活動では,新たに3つのアルゴリズムを導入した.改善率などの具体的な数値に関しては\ref{sec:theoryresult}に示しているので省略する.また昨年度までは実装されていなかったStage2を新たに適用した.理論班の検証結果によるとECMプログラムを最大で15\% ほど高速化することに成功した.また,広報班と協力しECMプログラムの使用法やアルゴリズムについて解説した.

\section{広報班}
アンケートの回答があまり集まっていないので,省略しました
\bunseki{橋本和典}

\end{document}

%まとめ
% 両面印刷する場合は `openany' を削除する
\documentclass[openany,11pt,papersize]{jsbook}

%パッケージの読み込みなど
% 報告書提出用スタイルファイル
\usepackage[final]{funpro}%最終報告書
%\usepackage[middle]{funpro}%中間報告書

% 画像ファイル (EPS, EPDF, PNG) を読み込むために
\usepackage[dvipdfmx]{graphicx,color}

%数式の表示に利用するため
\usepackage{amsmath,amssymb}

%アルゴリズムの表示に利用するパッケージ
\usepackage{algorithm}
\usepackage{algorithmic}

%枠をつけるためのパッケージ
\usepackage{ascmac}

%図の位置調整パッケージ
\usepackage{here}

%付録を作成するためのパッケージ
\usepackage{appendix}

%ドキュメント管理用パッケージ
\usepackage{docmute}


% ここから -->
\usepackage{calc,ifthen}
\newcounter{hoge}
\newcommand{\fake}[1]{\whiledo{\thehoge<70}{#1\stepcounter{hoge}}%
  \setcounter{hoge}{0}}
% <-- ここまで 削除してもよい


% 年度の指定
\thisYear{2016}

% プロジェクト名
\jProjectName{FUN-ECM プロジェクト}

% [簡易版のプロジェクト名]{正式なプロジェクト名}
% 欧文のプロジェクト名が極端に長い(2行を超える)場合は,短い記述を
% 任意引数として渡す.
%\eProjectName[Making Delicious curry]{How to make delicious curry of Hakodate}
\eProjectName{FUN-ECM Project}


% <プロジェクト番号>-<グループ名>
\ProjectNumber{15-A}

% グループ名
\jGroupName{Aグループ}
\eGroupName{A Group}

% プロジェクトリーダ
\ProjectLeader{1014129}{池野竜將}{Ryusuke Ikeno}

% グループリーダ
\GroupLeader  {1014129}{池野竜將}{Ryusuke Ikeno}

% メンバー数
\SumOfMembers{8}
% グループメンバ
\GroupMember  {1}{1014068}{駒ヶ嶺壮}{Sou Komagamine}
\GroupMember  {2}{1014109}{伊藤有輝}{Yuki Ito}
\GroupMember  {3}{1014129}{池野竜將}{Ryusuke Ikeno}
\GroupMember  {4}{1014137}{千葉大樹}{Daiju Chiba}
\GroupMember  {5}{1014164}{橋本和典}{Kazunori Hashimoto}
\GroupMember  {6}{1014168}{山下哲平}{Teppei Yamashita}
\GroupMember  {7}{1014209}{源啓多}{Keita Minamoto}
\GroupMember  {8}{1013150}{亀谷浩也}{Hiroya Kametani}

% 指導教員
\jadvisor{白勢政明,由良文孝}
% 複数人数いる場合はカンマ(,)で区切る.カンマの前後に空白は入れない.
\eadvisor{Masaaki Shirase, Fumitaka Yura}

% 論文提出日
\jdate{2016年7月27日}
\edate{July~27, 2016}


\begin{document}

\chapter{まとめ}

\section{前期活動結果}
前期は参考資料,論文,担当教員の白勢先生の講義による楕円曲線法の理解から始め,楕円曲線が楕円曲線法においていつどのように使われるかを理解した.その後,理論班,プロジェクト班の2班に分かれ作業を行った.理論班は,論文,入門書の読解をし,プログラム高速化のための改善案を出すことに成功した.しかし,前期中にプログラミング班が実装することはできなかった.プログラミング班は前年度のプロジェクトで作成されたECMプログラムを理解した.その後,実装ミスの改善や,新たなアルゴリズムの実装を行い,計算コストの減少に成功した.

\bunseki{千葉大樹}

\section{後期の展望}

後期は,理論班が作成したAtkin-Morain ECPPアルゴリズムを実装し,さらにECMプログラムの改善を図る.また,大きな合成数の分解を続けECMNETへのランクインを目指す.加えて,前期中に活動できなかった広報について新たに班を設置し活動していく.

\bunseki{橋本和典}

\section{後期活動結果}
後期はプログラミング,理論,広報にわかれ作業を始めた.プログラム班は初めに,前期中に理論班によって提案されたAtkin-Morain ECPPの実装を行った.その後は,スカラー倍算の高速化アルゴリズムを実装し,プログラムの高速化に成功した.また新たな試みとして今まで実装されていなかったStage2の実装を行ったが,効率はほとんど変わらなかった.理論班は完成したプログラムを検証するために,検証方法の調査・提案を行った.その後,自分たちで提案した検証方法を元に検証を行い,結果をまとめた.広報班では,最初にウェブページのコンテンツやターゲットについての提案を行った.その結果,メインターゲットを未来大を中心とする情報系の大学生とし,ECMについての基礎理論についてのページを作成することにした.またサブターゲットとして,来年のプロジェクトメンバー向けに今年度作成したプログラムについてのページを作成することにした.完成したウェブページはgh-pagesというサービスを利用して公開した.
\bunseki{池野竜將}

\section{全体を通して}
ECMを利用した素因数分解プログラムは,検証の結果分解する合成数の桁数が大きければ大きいほど改善しており,最大で15\%ほど高速化されている.しかし,発見できた合成数は112544281755782732673671367061(30)が最大であり,ECMNETへのランクインには最低でも64桁以上の素因数を発見する必要があるため,現状ではECMNETへのランクインは難しい.また,今年度の新しい活動として行ったFUN-ECMの広報活動は,アンケートによると理解出来た人とできない人に分かれたが,アンケートの解答数が少なく,評価はできなかった.
\bunseki{橋本和典}

\section{今後の課題と展望}
\begin{itemize}
\item 今までの試行で発見できた合成数は30桁が最大であり,ECMNETへのランクインは難しいと予想される.しかし,ECMは運要素の強いアルゴリズムの為,どのような原因で素因数が発見できていないかを理解していない.そのため,来年度は既に分解されている合成数の分解を並行して行い,その時点のプログラムでどれくらいの桁数の素因数を発見ができるかについても確認・検証が必要だと考えられる.
\item 広報作業開始が後期であったため,作成後に評価をするための時間が十分に取れなかった.そのため,来年度も広報活動を行うのであれば,前期から活動を開始し,学生からのフィードバックでを受ける時間を確保することが必要だと考えられる.
\end{itemize}

\bunseki{池野竜將}
\end{document}

%付録
% 両面印刷する場合は `openany' を削除する
\documentclass[openany,11pt,papersize]{jsbook}

%パッケージの読み込みなど
% 報告書提出用スタイルファイル
\usepackage[final]{funpro}%最終報告書
%\usepackage[middle]{funpro}%中間報告書

% 画像ファイル (EPS, EPDF, PNG) を読み込むために
\usepackage[dvipdfmx]{graphicx,color}

%数式の表示に利用するため
\usepackage{amsmath,amssymb}

%アルゴリズムの表示に利用するパッケージ
\usepackage{algorithm}
\usepackage{algorithmic}

%枠をつけるためのパッケージ
\usepackage{ascmac}

%図の位置調整パッケージ
\usepackage{here}

%付録を作成するためのパッケージ
\usepackage{appendix}

%ドキュメント管理用パッケージ
\usepackage{docmute}


% ここから -->
\usepackage{calc,ifthen}
\newcounter{hoge}
\newcommand{\fake}[1]{\whiledo{\thehoge<70}{#1\stepcounter{hoge}}%
  \setcounter{hoge}{0}}
% <-- ここまで 削除してもよい


% 年度の指定
\thisYear{2016}

% プロジェクト名
\jProjectName{FUN-ECM プロジェクト}

% [簡易版のプロジェクト名]{正式なプロジェクト名}
% 欧文のプロジェクト名が極端に長い(2行を超える)場合は,短い記述を
% 任意引数として渡す.
%\eProjectName[Making Delicious curry]{How to make delicious curry of Hakodate}
\eProjectName{FUN-ECM Project}


% <プロジェクト番号>-<グループ名>
\ProjectNumber{15-A}

% グループ名
\jGroupName{Aグループ}
\eGroupName{A Group}

% プロジェクトリーダ
\ProjectLeader{1014129}{池野竜將}{Ryusuke Ikeno}

% グループリーダ
\GroupLeader  {1014129}{池野竜將}{Ryusuke Ikeno}

% メンバー数
\SumOfMembers{8}
% グループメンバ
\GroupMember  {1}{1014068}{駒ヶ嶺壮}{Sou Komagamine}
\GroupMember  {2}{1014109}{伊藤有輝}{Yuki Ito}
\GroupMember  {3}{1014129}{池野竜將}{Ryusuke Ikeno}
\GroupMember  {4}{1014137}{千葉大樹}{Daiju Chiba}
\GroupMember  {5}{1014164}{橋本和典}{Kazunori Hashimoto}
\GroupMember  {6}{1014168}{山下哲平}{Teppei Yamashita}
\GroupMember  {7}{1014209}{源啓多}{Keita Minamoto}
\GroupMember  {8}{1013150}{亀谷浩也}{Hiroya Kametani}

% 指導教員
\jadvisor{白勢政明,由良文孝}
% 複数人数いる場合はカンマ(,)で区切る.カンマの前後に空白は入れない.
\eadvisor{Masaaki Shirase, Fumitaka Yura}

% 論文提出日
\jdate{2016年7月27日}
\edate{July~27, 2016}


\begin{document}

\appendix
\chapter{新規習得技術}

\begin{itemize}
\item PARI/GPの使用
\item Microsoft PowerPointの使用
\item Gitの使用
\item GitHubの使用
\item Xeno Phiの使用
\item functionviewの使用
\end{itemize}

\bunseki{橋本和典}

\chapter{相互評価}
ここでは,最終発表後の相互フィードバックで述べられたコメントを列挙する.

\begin{itemize}

\item 池野竜將
\begin{itemize}
\item リーダーとして,進捗管理,タスクの振り分け,発表資料の統合などを滞りなく行っていた.
\item プロジェクトリーダーとして一年間とても頼れる存在であった.メンバーのまとめ役や,スケジュールの管理,タスクの分担などリーダーとして様々な分野で活躍していた.
\item プロジェクトのリーダーとして,1年間プロジェクトの進行をしてくれた.日程調整や,役割分担,そのほかの作業に大きく関わってくれ,本プロジェクトのキーパーソンとして非常に活躍してくれた.
\item 1年間プロジェクト全体の統括をしてくれた.報告書やポスター等の全体で行う作業だけでなく,各班の進捗を確認するなど細かいところまで気配りができる頼れる存在だった.
\item 一年間プロジェクトリーダーとしての活動を責任持ってしてくれた.特にメンバー全員が効率よく活動できるよう日程調節と役割分担を毎回のプロジェクト時間までに考えてくれた.
\item プロジェクトリーダーとして責任をもって活動しており活動方針や予定を明確に表してくれて作業が滞ることなく進められるようにしてくれた.
\item プロジェクトリーダーとして,プロジェクトを先導していた.スケジュールの管理,作業状況の管理など活動をまとめるための重要な仕事を行っていた.
\end{itemize}

\item 源啓多
\begin{itemize}
\item 1年間通してプログラムの改善に取り組んでいた.また,プロジェクトを進行するうえで必要なツールをいち早く取り入れ使いこなしていた.人に教えるのは苦手と言いながらも他のプロジェクトメンバーに積極的にGitHubの使い方を教える等,活動の中心になっていた.
\item プログラム班として,自分たちが見つけたatkin-moraign ECPPの実装や,その他の実装など,ECMプログラムにおいて大きく貢献していた.彼の活躍のおかげで2016年度FUN-ECMはより大きな素因数を発見することができた.
\item 1年を通し,本プロジェクトの目的であるプログラムの改善を行っていた.その他にも,ウェブページ作成の際に使い慣れていないツールの使用方法を他メンバーに使い方を教えてあげていたりなど,他のメンバーの作業の手助けも積極的に行っていた.
\item プログミング班の中心としてECMプログラムの改善に大きく関わっていた.また,Githubやbootstrap等の導入に関して補助してくれたり,広報班の活動でも大きく貢献してくれた.
\item 一年間プログラムの製作に没頭してくれた.後期はプログラムだけではなく,検証班の検証方法についてもいろいろ教えてくれた.
\item 1年間のプロジェクトを通してプログラムに積極的に関わっていた.またほかの班にもプログラム使い方などを教えてくれたりしていた.
\item プログラマとして1年間プログラムの改善に努めていた.また,検証を行う上で必要な情報について教えてもらった.
\end{itemize}

\item 山下哲平
\begin{itemize}
\item 広報の作業では,インターンで学んだウェブページ作成の技術を生かし,伊藤・駒ヶ嶺と協力してスムーズに広報作業を進行させていた.前期に引き続きムードメーカー的な要素も持ちつつ,みんなが嫌がるような英訳などの地道な仕事も引き受け多方面からプロジェクトを支えていた.
\item ムードメーカーの役割を担いつつ,伊藤・駒ヶ嶺とWebサイトのデザインの改善・コンテンツの制作を積極的に行っていた.
\item 後期はウェブページ班として,活躍していた.もともとウェブデザインに興味を持っていたらしく,専門的な知識も周りのメンバーよりも多く持っていたため,ウェブページ作成の際にその知識を活用していた.
\item 後期では広報班の活動で私と協力して作業をしてくれた.共にwebページ制作に関して勉強し,スキルを高め合った.
\item 後期ではFUNECMで初めての活動内容である広報班のメンバーとしてwebページ製作を進めてくれた.
\item 後期からはウェブページの作成に取り組んでいた.またプロジェクトの場を盛り上げるなどし活動の雰囲気をよくしてくれていた.
\item 後期では,広報班としてウェブページの構成,コンテンツの作成を行っていた.ウェブページデザインについては前々からよく学んでいたらしく,ページ作成において重要なポジションを取っていた.
\end{itemize}

\item 伊藤有輝
\begin{itemize}
\item 広報の作業では,インターンで学んだウェブページ作成の技術を生かし,山下・駒ヶ嶺と協力してスムーズに広報作業を進行させていた.一度ウェブページが完成した後も,よくないと感じたところをすぐに提案し,自ら手直しを加える等積極的にブラッシュアップを行い,よりよいものを作成していた.
\item インターンで学んだ技術を活かし,山下・駒ヶ嶺とWebサイトのデザインの改善・コンテンツの制作を積極的に行っていた.
\item webページ作成において中心となって活動していた.自分とともに最初は試行錯誤していたが,最終的にスキルを高めあうことができた.たまにトイレに行っていた.
\item 後期はウェブページ班として活動を行っていた.もともと数学が好き,ということで,論文の読解の際,積極的に取り組んでいた.後期の活動の際もウェブページ作成の要として活動に積極的に取り組んでいた.
\item 広報班で山下,駒ヶ嶺と共にFUNECM活動発信してくれた.また発表前ではプレゼンの改善点をいろいろな方向から発言してくれた.
\item 広報班でウェブページの作成におり,ウェブページ以外にも発表資料の作成などにも活躍していた.
\item 広報班として活動を行い,ウェブページの作成を先導していた.プレゼン資料の作成も積極的に行っていた.
\end{itemize}

\item 駒ヶ嶺壮
\begin{itemize}
\item 後期から山下・伊藤とともに広報の作業を行っていた.あまり(ウェブページを作成する)技術はないと言っていたが,だからと言って受け身になるわけではなく,アイデア出しから紹介ページの作成など幅広く活動し,サポート役のような形で積極的に作業に参加していた.
\item 伊藤・山下とWebサイトのデザインの改善・コンテンツの制作を積極的に行っていた.
\item 伊藤,山下とともにwebページ作成において貢献した.また,前期も理論班として論文の解読において大きな貢献をしていた.発表や報告書作成においても大きな貢献をしていた.
\item 広報班の活動で共に作業した.webページ作成だけでなく,早い段階からポスターや報告書の作成に回り,チーム全体としての活動も進めてくれた.
\item 広報班で山下,伊藤と共にECMの活動発信してくれた.また,手の空いてる時は他のメンバーの協力も積極的にしてくれた.
\item 広報班として駒ヶ嶺と山下とでウェブページの作成に取り組んでいた.プレゼン発表の練習にもアドバイスなどしており活躍していた.
\item 広報班としてウェブページの作成を行っていた.また,プレゼン資料やポスターの作成も積極的に行い,発表の一助となった.
\end{itemize}

\item 橋本和典
\begin{itemize}
\item 後期では,検証を中心に行ってもらった.検証方法の調査から検証データのまとめまですべてに関わり,リーダーとして最後までやり切っていた.プログラムが改善されるたびに増える検証項目やまとめるデータ量に心を折ることなく仕事を全うしていた.
\item 亀谷・千葉と協力し,プログラムの速度がどれだけ改善されたかを検証し,まとめてくれた.
\item 理論班,検証班として前期は亀谷,千葉とともに前期は専門書の解読,後期はプログラムの検証で活躍してくれた.また,報告書やポスター,発表の際も活躍していた.
\item 後期は前期に引き続き理論班として活動を行っていた.プログラムの検証という難しい役だったが,他のメンバーに教えてもらったり,自力で調べたりなどし,積極的に活動に参加していた.
\item 理論班の活動として作成したプログラムを検証してくれた.検証では他のメンバーを先導し,検証の仕方から一生懸命取り組んでくれた.
\item 同じ班員として協力して成果を上げることができた,検証結果などを積極的にまとめてくれていた.
\item 検証班として,データの整理やまとめを積極的に行っていた.図表等で整理されたデータから分かったこと・考えられたことも多く,考察の時に役立った.良い検証班の先導役だった.
\end{itemize}

\item 亀谷浩也
\begin{itemize}
\item 前期に引き続き橋本のサポート役に徹していた.渡されたデータのまとめ・可視化や,検証活動を行っていた3人での意見交換の際に話を進める等,円滑な活動に必要不可欠だった.また,最終発表の際は,聴衆を意識した発表を行い企業の方からも評価されていた.
\item 橋本・千葉と協力し,プログラムの速度がどれだけ改善されたかを検証し,まとめてくれた.
\item 理論班,検証班として前期は橋本,千葉とともに前期は専門書の解読,後期はプログラムの検証で活躍してくれた.また,報告書やポスター,発表の際も活躍していた.
\item 後期は,千葉,橋本とともにプログラムの検証を行っていた.後期から始まった活動で,最終発表のスライドやポスターなどのグラフの作成なども最初からだったが,積極的に作成を行っていた.
\item 橋本と共にプログラムの検証を行っていた.検証以外の活動ではポスターの制作や最終スライドの準備に関して積極的に活動していた.
\item 検証班として私の意見を参考にしていろいろとアドバイスをくれた.また検証結果を可視化などの工夫をしてくれた.
\item 検証班として活動.報告書やポスター・スライドの作成に積極的に携わり,良い発表が行えるように努めていた.
\end{itemize}

\item 千葉大樹
\begin{itemize}
\item 検証作業に参加し,プログラムのデータをとることを中心に行っていた.プログラムを動かして,データを渡すという作業は多大な時間はかかるが,めんどくさがらずに行っていた.また,プレゼン資料のブラッシュアップの際には自分の意見をはっきりと伝え,よりよい方向に導いていた.
\item 橋本・亀谷と協力し,プログラムの速度がどれだけ改善されたかを検証し,まとめてくれた.
\item 理論班,検証班として前期は亀谷,橋本とともに前期は専門書の解読,後期はプログラムの検証で活躍してくれた.また,報告書やポスター,発表の際も活躍していた.
\item プログラムの検証班として積極的に検証を行っていた.もともと発表が苦手だったらしいのだが,一人で最終発表の練習を行っているなど,苦手を克服しようと積極的に活動していた.
\item プログラムの検証班として積極的に検証を行っていた.もともと発表が苦手だったらしいのだが,一人で最終発表の練習を行っているなど,苦手を克服しようと積極的に活動していた.プログラムの検証ではわからない点が出ると先生に聞きに行ったりなど問題を解決するために勢力的に活動していた.発表練習では前日に学校に残るなどグループ全体での成果に貢献していた.
\item 一年を通して,理論的な部分で困ってた時に理解した部分を教えてくれた.苦手なプレゼンもできるように練習をしていた.
\item プログラムの検証に主に関わってくれており,個人作業が多かったが検証などに取り組んでいた.
\end{itemize}

\end{itemize}

\bunseki{橋本和典}

\end{document}



%参考文献
% 両面印刷する場合は `openany' を削除する
\documentclass[openany,11pt,papersize]{jsbook}

%パッケージの読み込みなど
% 報告書提出用スタイルファイル
\usepackage[final]{funpro}%最終報告書
%\usepackage[middle]{funpro}%中間報告書

% 画像ファイル (EPS, EPDF, PNG) を読み込むために
\usepackage[dvipdfmx]{graphicx,color}

%数式の表示に利用するため
\usepackage{amsmath,amssymb}

%アルゴリズムの表示に利用するパッケージ
\usepackage{algorithm}
\usepackage{algorithmic}

%枠をつけるためのパッケージ
\usepackage{ascmac}

%図の位置調整パッケージ
\usepackage{here}

%付録を作成するためのパッケージ
\usepackage{appendix}

%ドキュメント管理用パッケージ
\usepackage{docmute}


% ここから -->
\usepackage{calc,ifthen}
\newcounter{hoge}
\newcommand{\fake}[1]{\whiledo{\thehoge<70}{#1\stepcounter{hoge}}%
  \setcounter{hoge}{0}}
% <-- ここまで 削除してもよい


% 年度の指定
\thisYear{2016}

% プロジェクト名
\jProjectName{FUN-ECM プロジェクト}

% [簡易版のプロジェクト名]{正式なプロジェクト名}
% 欧文のプロジェクト名が極端に長い(2行を超える)場合は,短い記述を
% 任意引数として渡す.
%\eProjectName[Making Delicious curry]{How to make delicious curry of Hakodate}
\eProjectName{FUN-ECM Project}


% <プロジェクト番号>-<グループ名>
\ProjectNumber{15-A}

% グループ名
\jGroupName{Aグループ}
\eGroupName{A Group}

% プロジェクトリーダ
\ProjectLeader{1014129}{池野竜將}{Ryusuke Ikeno}

% グループリーダ
\GroupLeader  {1014129}{池野竜將}{Ryusuke Ikeno}

% メンバー数
\SumOfMembers{8}
% グループメンバ
\GroupMember  {1}{1014068}{駒ヶ嶺壮}{Sou Komagamine}
\GroupMember  {2}{1014109}{伊藤有輝}{Yuki Ito}
\GroupMember  {3}{1014129}{池野竜將}{Ryusuke Ikeno}
\GroupMember  {4}{1014137}{千葉大樹}{Daiju Chiba}
\GroupMember  {5}{1014164}{橋本和典}{Kazunori Hashimoto}
\GroupMember  {6}{1014168}{山下哲平}{Teppei Yamashita}
\GroupMember  {7}{1014209}{源啓多}{Keita Minamoto}
\GroupMember  {8}{1013150}{亀谷浩也}{Hiroya Kametani}

% 指導教員
\jadvisor{白勢政明,由良文孝}
% 複数人数いる場合はカンマ(,)で区切る.カンマの前後に空白は入れない.
\eadvisor{Masaaki Shirase, Fumitaka Yura}

% 論文提出日
\jdate{2016年7月27日}
\edate{July~27, 2016}


\begin{document}

%\backmatter

% 参考文献
\begin{thebibliography}{9}

\bibitem{EN2016}
\newblock ECMNET.
\newblock https://members.loria.fr/PZimmermann/ecmnet/, (最終アクセス 2016年7月20日)

\bibitem{BD2013}
%{\ruby{藤沢}{ふじさわ}}{\ruby{幸穂}{ゆきほ}}.
\newblock Bernstein, D.J. , Birkner, P. , Lange, T. , Peters, C.
\newblock ECM USING EDWARDS CURVES.
\newblock Mathematics of Computation, 2013.

\bibitem{HH2008}
\newblock Hisil, H., Wong, K.K.-H., Carter, G., Dawson, E.
\newblock Twisted Edwards curves revisited.
\newblock Advances in Cryptology - ASIACRYPT 2008, 2008.

\bibitem{SK2016}
\newblock STUDIO KAMADA. 
\newblock http://stdkmd.com/, (最終アクセス 2016年7月15日).

\bibitem{JH2013}
\newblock Joseph H. S., John T.
\newblock 楕円曲線論入門
\newblock 丸善出版, 2012.

\bibitem{SCIS1997}
{\ruby{國廣}{くにひろ}}{\ruby{昇}{のぼる}},
 {\ruby{鶴岡}{つるおか}}{\ruby{行雄}{ゆきお}},
 {\ruby{小山}{ほり}}{\ruby{謙二}{けんじ}}.
\newblock 適切な位数を持つ楕円曲線に基づく素因数分解.
\newblock SCIS, 1997.

%後期に利用した論文

\bibitem{KG2006}
\newblock Kris Gaj, Soonhak Kwon, Patrick Baier, Paul Kohalbrenner, Hoang Le, Mohammed Khaleeluddin, Ramakrishna Bachimanchi.
\newblock Implementing th Elliptic Curve Methof of Factoring in Reconfogurable Hardware.
\newblock CHES-2006, 2006.

\bibitem{MT2015}
\newblock {\ruby{森下}{もりした}}{\ruby{拓也}{たくや}},Jibhui Chao.
\newblock 疑似的2次拡大環上の楕円曲線法.
\newblock FIT2015, 2015.

\bibitem{HH2016}
\newblock Henriette Heer, Gary McGuire, Oisin Robinson.
\newblock JKL-ECM: an implemention of ECM using Hessian curves.
\newblock LMS Journal of Computation and Mathmatcis, 2016.

\end{thebibliography}
\end{document}

\end{document}