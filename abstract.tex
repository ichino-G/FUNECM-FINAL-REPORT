% 両面印刷する場合は `openany' を削除する
\documentclass[openany,11pt,papersize]{jsbook}

%パッケージの読み込みなど
% 報告書提出用スタイルファイル
\usepackage[final]{funpro}%最終報告書
%\usepackage[middle]{funpro}%中間報告書

% 画像ファイル (EPS, EPDF, PNG) を読み込むために
\usepackage[dvipdfmx]{graphicx,color}

%数式の表示に利用するため
\usepackage{amsmath,amssymb}

%アルゴリズムの表示に利用するパッケージ
\usepackage{algorithm}
\usepackage{algorithmic}

%枠をつけるためのパッケージ
\usepackage{ascmac}

%図の位置調整パッケージ
\usepackage{here}

%付録を作成するためのパッケージ
\usepackage{appendix}

%ドキュメント管理用パッケージ
\usepackage{docmute}

% ここから -->
\usepackage{calc,ifthen}
\newcounter{hoge}
\newcommand{\fake}[1]{\whiledo{\thehoge<70}{#1\stepcounter{hoge}}%
  \setcounter{hoge}{0}}
% <-- ここまで 削除してもよい


% 年度の指定
\thisYear{2016}

% プロジェクト名
\jProjectName{FUN-ECM プロジェクト}

% [簡易版のプロジェクト名]{正式なプロジェクト名}
% 欧文のプロジェクト名が極端に長い(2行を超える)場合は,短い記述を
% 任意引数として渡す.
%\eProjectName[Making Delicious curry]{How to make delicious curry of Hakodate}
\eProjectName{FUN-ECM Project}


% <プロジェクト番号>-<グループ名>
\ProjectNumber{15-A}

% グループ名
\jGroupName{Aグループ}
\eGroupName{A Group}

% プロジェクトリーダ
\ProjectLeader{1014129}{池野竜將}{Ryusuke Ikeno}

% グループリーダ
\GroupLeader  {1014129}{池野竜將}{Ryusuke Ikeno}

% メンバー数
\SumOfMembers{8}
% グループメンバ
\GroupMember  {1}{1014068}{駒ヶ嶺壮}{Sou Komagamine}
\GroupMember  {2}{1014109}{伊藤有輝}{Yuki Ito}
\GroupMember  {3}{1014129}{池野竜將}{Ryusuke Ikeno}
\GroupMember  {4}{1014137}{千葉大樹}{Daiju Chiba}
\GroupMember  {5}{1014164}{橋本和典}{Kazunori Hashimoto}
\GroupMember  {6}{1014168}{山下哲平}{Teppei Yamashita}
\GroupMember  {7}{1014209}{源啓多}{Keita Minamoto}
\GroupMember  {8}{1013150}{亀谷浩也}{Hiroya Kametani}

% 指導教員
\jadvisor{白勢政明,由良文孝}
% 複数人数いる場合はカンマ(,)で区切る.カンマの前後に空白は入れない.
\eadvisor{Masaaki Shirase, Fumitaka Yura}

% 論文提出日
\jdate{2016年7月27日}
\edate{July~27, 2016}


\begin{document}

% 和文概要
\begin{jabstract}
 私たちのプロジェクトの目的は,より大きな桁数の素因数を見つけることである.素因数分解は,約30 年前から重要になってきている.その理由は,RSA 暗号にある.RSA 暗号は,安全性を2 つの大きな素因数からなる合成数の素因数分解が難しいことに依存している.しかし,技術の発展とともに素因数分解が従来よりも容易になってきてしまっているため,RSA 暗号が破られる可能性が高くなっている.そこで今注目されているのが楕円曲線暗号である.楕円曲線暗号は,RSA 暗号と同じ鍵長で高い安全性を保障することができる.そこで私たちは素因数分解をより簡単なものとすることで,RSA 暗号から楕円曲線暗号を主流とさせたい.
 私たちは,大きな素因数の発見のために,色々な文献を読んでその中から素因数分解を行うプログラムの改良法を発見する理論班と,それらの理論を利用して実際にプログラムの実装・改良を行い,プログラムを高速化させるプログラム班に分かれて活動を行った.
 理論班は,素因数分解がより高速に行われるようなアルゴリズムの発見を目標とした.楕円曲線法のプログラムは点の加算の繰り返しで行われるため,加算の計算コストを減らすことで高速な計算を可能とするための活動を行った.Atkin Morain ECPP を利用することで,従来の楕円曲線法よりも計算コストを削減できることを発見した.
 プログラム班は,前年度に作成された素因数分解プログラムをさらに高速化することを目標とした.前年度と同様に大きな数を扱うために,任意精度演算ライブラリのGMP を使用した.また,プログラムの並列実行を行うために,並列プログラムの為のAPI であるOpenMPを導入した.前年度に実装されたエドワーズ曲線よりも効率よく計算を行うため,extendedtwisted Edwards coordinates を実装した.同じ合成数に対してプログラムを実行する際の因数の発見確率をあげるために,Y の値をランダムに設定した.
 また,理論班とプログラム班で情報の交換を行ったり,協力を行ったりなど,2 つの班の活動により,素因数分解を高速に行うことができるプログラムが完成した.
 更に,今年度からの活動としてより多くの人にECMについて知ってもらうため,私たちFUN-ECMの活動内容を広報することにした.広報の方法として新たに広報班を結成し,簡単に閲覧できるようにWebページを作成することにした.閲覧するターゲットは主に情報系の大学生とした.
 広報班の活動により,半期で設定したターゲットに向けたWebページを作成することができた.

% 和文キーワード
\begin{jkeyword}
素因数分解,楕円曲線法,ECMNET,エドワーズ曲線,射影座標,RSA暗号
\end{jkeyword}
\bunseki{山下哲平}
\end{jabstract}

%英語の概要

\begin{eabstract}

~The goal of our project team is to find prime factor as large as possible. Factorizations in prime numbers have become more important since about thirteen years ago because the difficulty of factorization in prime numbers is related to Internet security. The reason lies in the RSA. The asymmetry of RSA is based on the practical difficulty of factoring the product of two large prime numbers. However, prime factorization is getting to easier by development in technology. Therefore, RSA is less secure compared to previously. That's why Elliptic Curve Cryptography (ECC) is paid more attention than RSA now. ECC ensure safety better than RSA cryptosystem with same key length.Accordingly, we make factorization in prime numbers simplify, we would like to change main cryptosystem from RSA cryptosystem to elliptic curve cryptography.

~In order to find prime factor as large as possible, we divided two groups, one is “theory group” that is to read various literature and to find algorithm of factorizations in prime numbers to calculate faster, the other is “programming group” that is to make program to base on algorithm.

 ~“Theory group” aims to find algorithm of factorizations in prime numbers to calculate faster. The ECM program repeats process of addition law many times over, therefore we reduced calculation. To access Atkin Morain construction, we were successful in calculation faster compared to previously.

~“Programming group” aim to improve program of last year project team faster than before. To treat large number likewise last year, we used arbitrary-precision arithmetic library called GMP. Also, we parallelize the program, we introduce Open MP is API for parallel program. We implement extended twisted Edwards coordinates efficient than Edwards curve implemented last year. Also, we set random Y’s value to raise finding assembly towards same composite numbers.
 
 We exchange information and cooperate “theory group” and “programming group”, we made a program that is to perform factorization in prime numbers fast.
 
 Furthermore, as activity from this fiscal year, we decided public relations activities of FUN-ECM. As method of public relations we formed public relation (P.R) group, and we decided to make webpage can browse easily. We established targets of browsing are university students of information system.
According to P.R group, we could make webpage directed at targets half year.
% 英文キーワード
\begin{ekeyword}
Elliptic Curve Method, prime factorization, ECMNET, Twisted Edwards Curve, Extended Twisted Edwards Coordinates, RSA cryptosystem
\end{ekeyword}
\bunseki{山下哲平}
\end{eabstract}

\end{document}

