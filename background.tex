% 両面印刷する場合は `openany' を削除する
\documentclass[openany,11pt,papersize]{jsbook}

%パッケージの読み込みなど
\input{settings.tex}

\begin{document}

\chapter{背景}

ECM(楕円曲線法)を利用した素因数分解は,実際のインターネットで使われる暗号技術の安全性の確認に必須であるため近年重要になってきており,それを利用しECM-NETにランクインすることが私たちの目的である.

\bunseki{駒ヶ嶺壮}

\section{本プロジェクトの背景}

現在インターネットを含む通信での鍵暗号技術(特に鍵共有と認証)においての主流はRSA暗号である.
RSA暗号とは公開鍵暗号の一つで,大きな合成数を素因数分解することの
難しさを安全性の根拠にした暗号である.
つまり,RSA暗号は公開鍵の整数を素因数分解できると解読可能となる.
楕円曲線法は数体篩法とともに最良な素因数分解法の1つである.

コンピューターの並列処理能力と計算能力の向上等で現在使用されている鍵でのRSA暗号方式は
近い将来解読される危険性が指摘されるようになった.
ここで,今後の暗号技術にはRSAに代わるものとして楕円曲線暗号(ECC)が注目され始めている.
ECCは現在の暗号技術において最も重要とされている手法である.
これは,暗号化・復号においてある有限体上の楕円曲線の点の加算を用いることにより,
RSA暗号と同じ鍵長でより解読が難しくなるからである.

以上のように,楕円曲線はRSA暗号への攻撃(つまり安全性解析)及びECCの構成に使用され,
両方とも社会的に重要性が高い.
更に興味深いことに,両者の主要演算は楕円曲線のスカラー倍算出るため,プログラムに類似点が多い.
私たちはECMに焦点をあて,FUN-ECMのECM-NETにランクインを
目指すことで函館から楕円曲線を利用する技術の重要性について発信することを目標として掲げた.

\bunseki{山下哲平}

\section{ECM-NETとは}

ECM-NETとは,楕円曲線法を用いて大きい桁数の素因数分解をみつけることを目的とした競争サイトである.
ECM-NETには現在登録されている素因数分解よりも大きな素因数を見つけることで誰でも
ランクインすることが可能である.
過去にランクインした日本人はK.Aoki氏とT.Izu氏の2名である.

\bunseki{駒ヶ嶺壮}

\section{課題の概要}\label{sec:gaiyou}

FUN-ECMがECM-NETへのランクインを目指すには大きい桁数の素因数を見つけなければいけないことから楕円曲線を用いた素因数分解のプログラムの並列処理と高速化を目指す.また,本プロジェクトの活動をWebサイト等を用いて外部に発信する.

\bunseki{駒ヶ嶺壮}

\end{document}
