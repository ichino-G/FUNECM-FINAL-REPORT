% 両面印刷する場合は `openany' を削除する
\documentclass[openany,11pt,papersize]{jsbook}

%パッケージの読み込みなど
% 報告書提出用スタイルファイル
\usepackage[final]{funpro}%最終報告書
%\usepackage[middle]{funpro}%中間報告書

% 画像ファイル (EPS, EPDF, PNG) を読み込むために
\usepackage[dvipdfmx]{graphicx,color}

%数式の表示に利用するため
\usepackage{amsmath,amssymb}

%アルゴリズムの表示に利用するパッケージ
\usepackage{algorithm}
\usepackage{algorithmic}

%枠をつけるためのパッケージ
\usepackage{ascmac}

%図の位置調整パッケージ
\usepackage{here}

%付録を作成するためのパッケージ
\usepackage{appendix}

%ドキュメント管理用パッケージ
\usepackage{docmute}

% ここから -->
\usepackage{calc,ifthen}
\newcounter{hoge}
\newcommand{\fake}[1]{\whiledo{\thehoge<70}{#1\stepcounter{hoge}}%
  \setcounter{hoge}{0}}
% <-- ここまで 削除してもよい


% 年度の指定
\thisYear{2016}

% プロジェクト名
\jProjectName{FUN-ECM プロジェクト}

% [簡易版のプロジェクト名]{正式なプロジェクト名}
% 欧文のプロジェクト名が極端に長い(2行を超える)場合は,短い記述を
% 任意引数として渡す.
%\eProjectName[Making Delicious curry]{How to make delicious curry of Hakodate}
\eProjectName{FUN-ECM Project}


% <プロジェクト番号>-<グループ名>
\ProjectNumber{15-A}

% グループ名
\jGroupName{Aグループ}
\eGroupName{A Group}

% プロジェクトリーダ
\ProjectLeader{1014129}{池野竜將}{Ryusuke Ikeno}

% グループリーダ
\GroupLeader  {1014129}{池野竜將}{Ryusuke Ikeno}

% メンバー数
\SumOfMembers{8}
% グループメンバ
\GroupMember  {1}{1014068}{駒ヶ嶺壮}{Sou Komagamine}
\GroupMember  {2}{1014109}{伊藤有輝}{Yuki Ito}
\GroupMember  {3}{1014129}{池野竜將}{Ryusuke Ikeno}
\GroupMember  {4}{1014137}{千葉大樹}{Daiju Chiba}
\GroupMember  {5}{1014164}{橋本和典}{Kazunori Hashimoto}
\GroupMember  {6}{1014168}{山下哲平}{Teppei Yamashita}
\GroupMember  {7}{1014209}{源啓多}{Keita Minamoto}
\GroupMember  {8}{1013150}{亀谷浩也}{Hiroya Kametani}

% 指導教員
\jadvisor{白勢政明,由良文孝}
% 複数人数いる場合はカンマ(,)で区切る.カンマの前後に空白は入れない.
\eadvisor{Masaaki Shirase, Fumitaka Yura}

% 論文提出日
\jdate{2016年7月27日}
\edate{July~27, 2016}


\begin{document}

\chapter{背景}

ECM(楕円曲線法)を利用した素因数分解は近年重要になっており,それを利用しECM-NETにランクインすることが私たちの目的である.

\bunseki{駒ヶ嶺壮}

\section{本プロジェクトの背景}

現在インターネットを含む通信での暗号技術においての主流はRSA暗号である.RSA暗号とは公開鍵暗号の一つで,大きな合成数を素因数分解することの難しさを安全性の根拠にした暗号である.しかし,スーパーコンピューターの並行処理能力と計算能力の向上等で鍵長1024ビットのRSA暗号方式は解読される危険性が指摘されるようになった.ここで,今後の暗号技術にはRSAに変わるものとして楕円曲線暗号が使われて始めている.楕円曲線暗号は現在の暗号技術において最も重要とされている手法である.これは,ある楕円曲線における有限体上の楕円曲線の点の加算を用いることにより,RSA暗号と同じ鍵長でより解読が難しくなるからである.ここで私たちはこの楕円曲線暗号の中で核となる楕円曲線を用いた素因数分解のアルゴリズムについて考え,FUN-ECMがECM-NETにランクインを目指すことで函館から楕円曲線,素因数分解,暗号技術の重要性について発信することを目標として掲げた.

\bunseki{山下哲平}

\section{ECM-NETとは}

ECM-NETとは,楕円曲線法を用いて大きい桁数の素因数を見つけることを目的とした競争サイトである.ECM-NETには現在登録されている素因数分解よりも大きな素因数を見つけることで誰でもランクインすることが可能である.

\bunseki{駒ヶ嶺壮}

\section{課題の概要}\label{sec:gaiyou}

FUN-ECMがECM-NETへのランクインを目指すには大きい桁数の素因数を見つけなければいけないことから楕円曲線を用いた素因数分解のプログラムの並列処理と高速化を目指す.また,本プロジェクトの活動をWebサイト等を用いて外部に発信する.

\bunseki{駒ヶ嶺壮}

\end{document}
