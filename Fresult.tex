% 両面印刷する場合は `openany' を削除する
\documentclass[openany,11pt,papersize]{jsbook}

%パッケージの読み込みなど
% 報告書提出用スタイルファイル
\usepackage[final]{funpro}%最終報告書
%\usepackage[middle]{funpro}%中間報告書

% 画像ファイル (EPS, EPDF, PNG) を読み込むために
\usepackage[dvipdfmx]{graphicx,color}

%数式の表示に利用するため
\usepackage{amsmath,amssymb}

%アルゴリズムの表示に利用するパッケージ
\usepackage{algorithm}
\usepackage{algorithmic}

%枠をつけるためのパッケージ
\usepackage{ascmac}

%図の位置調整パッケージ
\usepackage{here}

%付録を作成するためのパッケージ
\usepackage{appendix}

%ドキュメント管理用パッケージ
\usepackage{docmute}


% ここから -->
\usepackage{calc,ifthen}
\newcounter{hoge}
\newcommand{\fake}[1]{\whiledo{\thehoge<70}{#1\stepcounter{hoge}}%
  \setcounter{hoge}{0}}
% <-- ここまで 削除してもよい


% 年度の指定
\thisYear{2016}

% プロジェクト名
\jProjectName{FUN-ECM プロジェクト}

% [簡易版のプロジェクト名]{正式なプロジェクト名}
% 欧文のプロジェクト名が極端に長い(2行を超える)場合は,短い記述を
% 任意引数として渡す.
%\eProjectName[Making Delicious curry]{How to make delicious curry of Hakodate}
\eProjectName{FUN-ECM Project}


% <プロジェクト番号>-<グループ名>
\ProjectNumber{15-A}

% グループ名
\jGroupName{Aグループ}
\eGroupName{A Group}

% プロジェクトリーダ
\ProjectLeader{1014129}{池野竜將}{Ryusuke Ikeno}

% グループリーダ
\GroupLeader  {1014129}{池野竜將}{Ryusuke Ikeno}

% メンバー数
\SumOfMembers{8}
% グループメンバ
\GroupMember  {1}{1014068}{駒ヶ嶺壮}{Sou Komagamine}
\GroupMember  {2}{1014109}{伊藤有輝}{Yuki Ito}
\GroupMember  {3}{1014129}{池野竜將}{Ryusuke Ikeno}
\GroupMember  {4}{1014137}{千葉大樹}{Daiju Chiba}
\GroupMember  {5}{1014164}{橋本和典}{Kazunori Hashimoto}
\GroupMember  {6}{1014168}{山下哲平}{Teppei Yamashita}
\GroupMember  {7}{1014209}{源啓多}{Keita Minamoto}
\GroupMember  {8}{1013150}{亀谷浩也}{Hiroya Kametani}

% 指導教員
\jadvisor{白勢政明,由良文孝}
% 複数人数いる場合はカンマ(,)で区切る.カンマの前後に空白は入れない.
\eadvisor{Masaaki Shirase, Fumitaka Yura}

% 論文提出日
\jdate{2016年7月27日}
\edate{July~27, 2016}


\begin{document}

\chapter{前期活動成果}
本プロジェクトでは,理論班で理解することに成功した高速化アルゴリズムをプログラミング班に伝え,プログラミング班がそのアルゴリズムを実装することによりECMプログラムを作成した.

\bunseki{千葉大樹}

\section{理論班}
理論班は,活動内容で示した射影座標を用いたスカラー倍算楕円曲線プログラムにおける変数の点の与え方のアルゴリズムを発見した.これにより乗算の回数,除算の回数が減少したことにより素因数を発見する効率を理論上1.5 倍に上昇させることができる。しかし、実装前との計算コストの実数値の比較をすることはできなかった.また,Atkin-Moraine ECPP アルゴリズムの理解に成功した.このアルゴリズムを実装することにより,位数があらかじめ小さな因数d を持つ曲線のみを使用し,素因数をp とした場合,ランダムに動くサイズがp からp/d に減少するため因数分解に成功する確率を高めることができる\cite{SCIS1997}.しかし,まだ実装には至っていないため,前期の活動はアルゴリズムの読解についてまとめたレポートを作成し,論文の読解を終了した.

\bunseki{駒ヶ嶺壮}

\section{プログラミング班}
プログラミング班では,新たなアルゴリズムを実装し,理論上は\ref{tab:cost}のように計算量が減少することが分かった.詳細な実験は行っておらず有意な差があるかどうかは確認できていない.だが,実際に素因数分解を行った結果,処理が早くなっていることが確認できた.

\begin{table}
\begin{center}
\caption{昨年度と今年度のプログラムの計算コストの比較}\label{tab:cost}
\begin{tabular}{ccc}
\hline
& 2倍算 & 2倍算→加算\\
\hline
昨年度 & 3{\bf M}+4{\bf S}+1{\bf D}\footnotemark & 13{\bf M}+5{\bf S}+3{\bf D}\\
今年度 & 3{\bf M}+4{\bf S}+1{\bf D} & 12{\bf M}+4{\bf S}+1{\bf D}\\
\hline
\end{tabular}
\end{center}
\end{table}
\footnotetext{{\bf M}:乗算,{\bf S}:2乗算,{\bf D}:楕円曲線の係数a,dを用いた乗算}

また,実際に巨大な合成数を分解し,昨年度のプログラムとの性能を比較することにした.評価するにあたって,2015年度に作成されたプログラムでテストに使用されていた合成数$10^{306}+1$を素因数分解することで,以前のプログラムとの比較をすることとした.2015年度のプログラムでこの合成数を分解した結果,発見されたもっとも大きな素因数は157538980319816607(21桁)であった.同様に今年改善されたプログラムで分解した結果,発見されたもっとも大きな素因数は112544281755782732673671367061(30桁)であり,より大きな素因数を見つけることができるように改善された.素因数が見つかった際のログを下に示す.
 
\begin{itembox}[H]{30桁発見の際のログ}
Stage1: d = 126909574787277066813799168682279610402560329810862501906041

Stage1 time: 5.359253 seconds

Stage2 time: -----

total time: 5.359 seconds

Y=46730248666831195065236836785868114479777957743528058131679

--------------------------------------------------

@ probable prime factor found: 184736584265492707905284574931  digits: 30 cofactor: 738759178437819643189478148923
\end{itembox}

\bunseki{源啓多}

\end{document}